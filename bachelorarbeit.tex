\documentclass[a4paper,index=totoc,toc=bibliography,fontsize=12,DIV=13,headinclude,twoside,BCOR=12mm,cleardoublepage=empty,headsepline=1pt,draft]{scrreprt}
\usepackage{./config/preamble}
%\hfuzz=500pt %ignorieren von underfulll hbox

%Titel vom Inhaltsverzeichnis ändern
\addto\captionsngerman{% Replace "ngerman" with the language you use
  \renewcommand{\contentsname}%
    {Contents}%
}

\begin{document}
\pagenumbering{Roman} % Seitennummerierung auf römische Zahlen setzen
\begin{titlepage}
	% Nach einer Vorlage von http://www.LaTeXTemplates.com
	\newcommand{\HRule}{\rule{\linewidth}{0.5mm}} % Defines a new command for the horizontal lines, change thickness here

	\center % Center everything on the page
 

	\textsc{\LARGE Westfälische Wilhelms-Universität Münster}\\[1.5cm] % Name of your university/college
	\textsc{\Large Bachelor thesis}\\[0.5cm] % Major heading such as course name


	\HRule \\[0.4cm]
	
	{\onehalfspacing\huge\sffamily\bfseries Monads, Comonads and Witt Vectors \singlespacing} % Title of your document
	\vspace{-0.4cm}
	\HRule \\[1.5cm] 
	

	\begin{minipage}[t]{0.4\textwidth}
	\begin{flushleft} \large
	\emph{Author:}\\
	Jakob \textsc{Bernd}\\ % Your name
	\normalsize \url{jbernd@uni-muenster.de}\\
	Matr.\,Nr. 514640
	\end{flushleft}
	\end{minipage}
	~
	\begin{minipage}[t]{0.4\textwidth}
	\begin{flushright} \large
	\emph{Supervisor:} \\
	Prof. Dr. Christopher \textsc{Deninger}\\ % Supervisor's Name
	\end{flushright}
	\end{minipage}\\[4cm]

	{\large submitted on \today}\\[3cm] % Date, change the \today to a set date if you want to be precise

	\includegraphics[width=0.2\textwidth]{resources/WWU_Logo.png}
	\includegraphics[height=1.3cm,keepaspectratio]{resources/fb10logo.pdf}\\[1cm] % Include a department/university logo - this will require the graphicx package
	

	\vfill % Fill the rest of the page with whitespace
	
\end{titlepage}
\begin{abstract}
\section*{Preface}
In this bachelor thesis, I will give an introduction to the concept of monads,
which are a generalization of monoids, and their categorical duals, comonads.
Every adjunction induces a monad, but every monad is induced by an adjunction as well.
In the third and last section, will also introduce Witt vectors, which are used a lot in
algebraic number theory and proof that they admit a comonad structure. At last,
I will show that the teichmüller map induces a morphism of comonads between
the free monoid ring comonad and the witt vector comonad.
\end{abstract}
\tableofcontents
\cleardoubleoddemptypage
\pagenumbering{arabic}
\setcounter{page}{1}


\chapter{Adjoint situations}

\todo{einleitender Satz}

\begin{proposition}
    Given two functors
    \begin{tikzcd}
        \cat{A} \ar[r,"G",shift left = .60ex]
          & 
        \cat{B} \ar[l,"F",shift left = .60ex]
    \end{tikzcd}
    the following are equivalent: 
    \begin{enumerate}[(a)]
        \item $\exists \eta\colon \id_B \Rightarrow GF$ and $\eps\colon FG \Rightarrow \id_A$ 
        natural transformations such that $\forall a \in Ob(A), b \in Ob(B)$ 
        the following two diagrams commute:
        \begin{equation}
            \begin{tikzcd}\tag{triangle identity}
                A \arrow[rd,"\id_{F(b)}"'] \arrow[r, "F(\eta_b)"] & B \arrow[d,"\eps_{F(b)}"] \\
                                            & C
            \end{tikzcd}
            \qqq
            \begin{tikzcd} 
                A \arrow[rd,"\id_{F(b)}"'] \arrow[r, "F(\eta_b)"] & B \arrow[d,"\eps_{F(b)}"] \\
                                            & C
            \end{tikzcd}
        \end{equation}
        \item $\forall a \in Ob(A), b \in Ob(B)$ there is a bijection 
        \[
        \phi_{a,b}\colon\Hom[A](F(b),a)\to \Hom[B](b,G(a))
        \]
        which is natural in a and b, i.e. for $p\colon a\to a':$
        \[
            \begin{tikzcd}
                \Hom[A](F(b),a) \arrow[r] \arrow[d] 
                  & \Hom[B](b,G(a)) \arrow[d] \\
                \Hom[A](F(b),a') \arrow[r]
                  & \Hom[B](b,G(a'))
            \end{tikzcd}
        \]
        and for $q\colon b \to b':$
        \[
            \begin{tikzcd}
                \Hom[A](F(b'),a) \arrow[r] \arrow[d] 
                  & \Hom[B](b',G(a)) \arrow[d] \\
                \Hom[A](F(b),a) \arrow[r]
                  & \Hom[B](b,G(a))
            \end{tikzcd}
        \]
    \end{enumerate}
\end{proposition}
\begin{beweis}
    $(a)\implies (b):$ \\
    define 
    \[
        \phi_{a,b}\colon \Hom(F(b),a)\to \Hom(b,G(a))
    \] by $g \mapsto G(g) \circ \eta_b $
    for $g\colon F(b)\to a$
\end{beweis}


\chapter{Monads and Comonads}
\section{Definition of Monads and Comonads}
A central notion in algebra is that of a \textit{monoid},
that is, a set $M$ equipped with a map 
\todo{definitionen nicht nummerieren?}
$\mu \colon M \times M \to M$; $(a,b) \mapsto a \cdot b$ 
(often called \textit{multiplication}) and an element $e \in M$
such that the following two axioms hold:
\begin{align}
    \label{eq: associativity for a monoid}  \tag{associativity} 
    (a \cdot b) \cdot c = a \cdot (b \cdot c) 
    \quad &\text{for all} \ a,b,c \in M. \\
    \label{eq: identity element for a monoid} \tag{identity element}
    e \cdot a = a \cdot e = a \quad &\text{for all} \ a \in M 
\end{align}
We can give an equivalent definition in terms of maps and commuting diagrams as follows:
A \textit{monoid} is a set $M$ together with two functions 
\[
    \mu \colon M \times M \to M, \quad 
    e \colon \pt \to M
\]
such that the following diagrams commute: \\

\begin{figure}[H]
    \centering
    \begin{subfigure}{0.4\textwidth}
    \centering
    %\phantomsection\label{name1}
        \begin{tikzcd}
            M \times M \times M \ar[r,"\id \times \mu"] \ar[d,"\mu \times \id"] 
              & M \times M \ar[d,"\mu"] \\
            M \times M \ar[r,"\mu"]
              & M
        \end{tikzcd}
    %\caption*{(cap1)}
    \end{subfigure}
    \hspace{2em}
    \begin{subfigure}{0.4\textwidth}
    \centering
    %\phantomsection\label{name2}
        \begin{tikzcd}
            \pt \times M \ar[rd,"l"'] \ar[r, "e \times \id"] 
            & M \times M \ar[d,"\mu"] 
            & M \times \pt \ar[l,"\id \times e"'] \ar[ld,"r"]\\
            & M
        \end{tikzcd}
    %\caption*{(cap2)}
    \end{subfigure}
    \end{figure}


where $\id$ is the identity on m, and $l$ and $r$ are the canonical bijections
\begin{align*}
    &l \colon \pt \times M \to M;\ l(\ast,m) = m \\
    &r \colon M \times \pt \to M;\ r(m,\ast) = m.
\end{align*}

Explicitly, the first diagram means that for all $a,b,c \in M$:
\[
    (a \cdot b) \cdot c = a \cdot (b \cdot c) 
    \quad \text{for all} \ a,b,c \in M.
\]
which is verbatim the~\ref{eq: associativity for a monoid} axiom, the second diagram means that for all $m \in M$:
\[
  e(\ast) \cdot m = l(\ast,m) = m = r(m,\ast) = m \cdot e(\ast)  
\]
which is clearly the~\ref{eq: identity element for a monoid} axiom 
for the element $e(\ast)$.
This motivates the following definition: \todo{monoid/monad/ monoid object}
\begin{definition}[Monad]
A \textit{Monad} $(T,\mu, \eta) $ in a category $\mathcal{X}$ consists of
\begin{itemize}
    \item an endofunctor $T\colon \mathcal{X} \to \mathcal{X}$
    \item a natural transformation $\eta \colon \id_\mathcal{X} \Rightarrow T$ 
    \item a natural transformation $\mu\colon T^2 \Rightarrow T $
\end{itemize}  
such that the following diagrams commute: \\
% der ' nach dem Text ändert die Position der Pfeil-Beschriftung
\begin{figure}[H]
    \centering
    \begin{subfigure}{0.3\textwidth}
        \centering
        \phantomsection\label{dia: associativity}
        % Content of the first subfigure
        \begin{tikzcd}
            T^3 \ar[r,"T\mu",Rightarrow] \ar[d,"\mu T"',Rightarrow] 
            & T^2 \ar[d,"\mu",Rightarrow] \\
            T^2 \ar[r,"\mu",Rightarrow]
            & T
        \end{tikzcd}
        \caption*{(associativity)}
    \end{subfigure}
    \hspace{2em}
    \begin{subfigure}{0.3\textwidth}
        \centering
        \phantomsection\label{dia: unitality}
        \begin{tikzcd}
            T \ar[rd,"\id_T"',Rightarrow] \ar[r, "\eta T",Rightarrow] 
        & T^2 \ar[d,"\mu",Rightarrow] 
        & T \ar[l,"T \eta"',Rightarrow] \ar[ld,"\id_T",Rightarrow]\\
        & T
        \end{tikzcd}
        \caption*{(unitality)}
    \end{subfigure} 
\end{figure}

In terms of components,~\refassociativity and~\refunitality mean that for every object $x$ of $\mathcal{X}$
the following diagrams commute:

\begin{figure}[H]
\centering
\begin{subfigure}{0.4\textwidth}
\centering
%\phantomsection\label{name1}
\begin{tikzcd}
    T(T(Tx)) \ar[r,"T(\mu_x)"] \ar[d,"\mu_{Tx}"'] 
    & T(Tx) \ar[d,"\mu_x"] \\
    T(Tx) \ar[r,"\mu_x"]
    & Tx
\end{tikzcd}
\caption*{(associativity)}
\end{subfigure}
\hspace{2em}
\begin{subfigure}{0.4\textwidth}
\centering
%\phantomsection\label{name2}
\begin{tikzcd}
    Tx \ar[rd,"\id_{Tx}"'] \ar[r, "\eta_{Tx}"] 
    & T(Tx) \ar[d,"\mu_x"] 
    & Tx \ar[l,"T(\eta_x)"'] \ar[ld,"\id_{Tx}"]\\
    & Tx
\end{tikzcd}
\caption*{(unitality)}
\end{subfigure}
\end{figure}

\end{definition}

\begin{example}[preorder]
Recall: A \textit{preorder} $(\mathcal{P},\le)$ is a category with $\mathcal{P}$ as objects and 
a morphism between $X$ and $Y$ iff $X \le Y$.
A functor $T\colon \mathcal{P} \to \mathcal{P}$ is thus a monotonic function $\mathcal{P}\to \mathcal{P}$
($x \le y \implies Tx\le Ty$).
The existence of the natural transformations $\eta$ is equivalent to
\[x \le Tx \ \forall x \in \mathcal{P}\]
and the existence of $\mu$ is equivalent to
\[T(Tx) \le Tx \ \forall x \in \mathcal{P}\] 
because there is at most one morphism $x \to y$, so the neccessary diagrams commute trivially.\\
Now suppose $\mathcal{P}$ is a \textit{partial order}, i.e. $x \le y \le x \implies x = y \ \forall x,y \in \mathcal{P}$. \\
Then:
\begin{align*}
    x \le Tx \implies Tx \le T(Tx) \\
    T(Tx) \le Tx \implies Tx = T(Tx)
\end{align*}
so a Monad $T$ in a partial order $\mathcal{P}$ is a \textit{closure operation} in $\mathcal{P}$, i.e. 
a monotonic function $T \colon \mathcal{P} \to \mathcal{P}$ 
with $x \le Tx$ and $T(Tx)=Tx \ \forall x \in \mathcal{P}.$ \\
Now every topological space $X$ induces a partial order $\mathcal{P} = (\mathscr{P}(X),\subseteq)$.
Here an example for a closure operation is taking the topological closure $A \mapsto \overline{A}$,
since it holds for all $A \subseteq X$ that $A \subseteq \overline{A}$ and
$\overline{\overline{A}} = \overline{A}$.
\end{example}

\begin{definition}[Comonad]
A \textit{Comonad} $(L,\eps, \omega) $ in a Category $\mathcal{A}$ consists of
\begin{itemize}
    \item an endofunctor $L\colon \mathcal{A} \to \mathcal{A}$
    \item a natural transformation $\eps \colon L \Rightarrow \id_{\mathcal{A}}$ 
    \item a natural transformation $\omega\colon L \Rightarrow L^2 $
\end{itemize}  
such that the following diagrams commute:

\begin{figure}[H]
\centering
\begin{subfigure}{0.4\textwidth}
\centering
%\phantomsection\label{name1}
    \begin{tikzcd}
        L \ar[r,"L\omega",Rightarrow] \ar[d,"\omega L"',Rightarrow] 
            & L^2 \ar[d,"L\omega",Rightarrow] \\
        L^2 \ar[r,"\omega L",Rightarrow]
            & L^3
    \end{tikzcd}
\caption*{(counitality)}
\end{subfigure}
\hspace{2em}
\begin{subfigure}{0.4\textwidth}
\centering
%\phantomsection\label{name2}
    \begin{tikzcd} 
        & L \ar[ld,"\id_L"',Rightarrow] 
        \ar[rd,"\id_L",Rightarrow] \ar[d,"\omega",Rightarrow] & \\
        L 
        & L^2 \ar[l,"\eps L"',Rightarrow] \ar[r,"L \eps",Rightarrow] 
        & L
    \end{tikzcd}
\caption*{(coassociativity)}
\end{subfigure}
\end{figure}


In terms of components, this means that for every object $x$ of $\mathcal{A}$
the following diagrams commute:

\begin{figure}[H]
\centering
\begin{subfigure}{0.4\textwidth}
\centering
%\phantomsection\label{name1}
    \begin{tikzcd}
        Lx \ar[r,"L(\omega_x)"] \ar[d,"\omega_{Lx}"'] 
            & L(Lx) \ar[d,"L(\omega_x)"] \\
        L(Lx) \ar[r,"\omega_{Lx}"]
            & L(L(Lx))
    \end{tikzcd}
\caption*{(counitality)}
\end{subfigure}
\hspace{2em}
\begin{subfigure}{0.4\textwidth}
\centering
%\phantomsection\label{name2}
    \begin{tikzcd}
        & Lx \ar[ld,"\id_{Lx}"'] 
        \ar[rd,"\id_{Lx}"] \ar[d,"\omega_x"] & \\
        Lx 
        & L(Lx) \ar[l,"\eps_{Lx}"'] \ar[r,"L(\eps_x)"] 
        & Lx
    \end{tikzcd}
\caption*{(coassociativity)}
\end{subfigure}
\end{figure}

\end{definition}

\begin{lemma}
    For every object $x$ in $\mathcal{X}$, the following diagram commutes:
    \[
      \begin{tikzcd}
        T(Tx) \arrow[r,"T(\delta_x)"] \arrow[d,"\delta_{Tx}"] 
            & T(T'x) \arrow[d,"\delta_{T'x}"] \\
          T(T'x) \arrow[r,"T'(\delta_x)"]
            & T'(T'x)
      \end{tikzcd}
    \]
    this means \[
        \delta T' \circ T \delta = T' \delta \circ \delta T
        \colon T^2 \Rightarrow (T')^2.
    \]
\end{lemma}
\begin{beweis}
    $\delta_x \colon Tx \to T'x$ is a ring homomorphism.
    Since $\delta \colon T \Rightarrow T'$ is natural transformation, the square commutes.
\end{beweis}
\begin{definition}[Morphism of monads]
    Let $\mathcal{X}$ be a category, let $(T,\eta,\mu)$ and $(T',\eta',\mu')$ be monads in $\mathcal{X}$.
    We say that a natural transformation $\nat[\delta]{T}{T'}$ is a \textit{morphism of monads} if it preserves
    the unit and the multiplication, i.e. the following diagrams commute:

    \begin{figure}[H]
    \centering
    \begin{subfigure}{0.4\textwidth}
    \centering
    %\phantomsection\label{name1}
    \begin{tikzcd}
        \id_T \ar[rd,"\eta'"',Rightarrow] \ar[r, "\eta",Rightarrow] 
        & T \ar[d,"\delta",Rightarrow] \\
        & T'
    \end{tikzcd}
    \caption*{(unit-preserving)}
    \end{subfigure}
    \hspace{2em}
    \begin{subfigure}{0.4\textwidth}
    \centering
    %\phantomsection\label{name2}
    \begin{tikzcd}
        T^2 \ar[r,"\mu",Rightarrow] \ar[d,"\delta T' \circ T\delta"',Rightarrow] 
        & T \ar[d,"\delta",Rightarrow] \\
        T'^2 \ar[r,"\mu'",Rightarrow]
        & T'
    \end{tikzcd}
    \caption*{(multiplication-preserving)}
    \end{subfigure}
    \end{figure}

    %vadjust allows todo in math-mode   
\end{definition}
\begin{definition}[Morphism of comonads]
    Let $\mathcal{A}$ be a category, let $(L,\eps,\omega)$ and $(L',\eps',\omega')$ be comonads in $\mathcal{A}$.
    We say that a natural transformation $\nat[\delta]{L}{L'}$ is a \textit{morphism of monads} if it preserves
    the counit and the comultiplication, i.e. the following diagrams commute:
    \begin{figure}[H]
    \centering
    \begin{subfigure}{0.4\textwidth}
    \centering
    %\phantomsection\label{name1}
    \begin{tikzcd}
        L \ar[r,"\delta",Rightarrow] \ar[rd,"\eps"',Rightarrow]
        & L' \ar[d,"\eps'",Rightarrow] \\
        & \id_A 
    \end{tikzcd}
    \caption*{(counit-preserving)}
    \end{subfigure}
    \hspace{2em}
    \begin{subfigure}{0.4\textwidth}
    \centering
    %\phantomsection\label{name2}
    \begin{tikzcd}
        L \arrow[r,"\omega",Rightarrow] \arrow[d,"\delta",Rightarrow] 
          & L^2 \arrow[d,"\delta L' \circ L\delta",Rightarrow] \\
        L' \arrow[r,"\omega'",Rightarrow]
          & {L'}^2
    \end{tikzcd}
    \caption*{(comultiplication-preserving)}
    \end{subfigure}
    \end{figure}
\end{definition}

\section{The Eilenberg-Moore-Category of a Monad}
This section will answer the question, whether every Monad is induced by an adjunction.

\begin{definition}[Eilenberg-Moore-Category]
    Let $T = (T,\eta,\mu)$ be a monad in a category $\cat{X}$.
    A \textit{$T$-algebra} is a pair $(x,h)$ where $x$ is an object of $\cat{X}$ and $h \colon Tx \to x$ is 
    an arrow such that the following diagrams commute:
    \begin{figure}[H]
    \centering
    \begin{subfigure}{0.4\textwidth}
    \centering
    %\phantomsection\label{name1}
    \begin{tikzcd}
        T^2x \arrow[r,"Th"] \arrow[d,"\mu"] 
          & Tx \arrow[d,"h"] \\
        T \arrow[r,"h"]
          & x
    \end{tikzcd}
    %\caption*{(cap1)}
    \end{subfigure}
    \hspace{2em}
    \begin{subfigure}{0.4\textwidth}
    \centering
    %\phantomsection\label{name2}
    \begin{tikzcd}
        x \ar[r,"\eta_x"] \ar[rd,"id_x"']
        & Tx \ar[d,"h"] \\
        & x 
    \end{tikzcd}
    %\caption*{(cap2)}
    \end{subfigure}
    \end{figure}
    We call $h$ the \textit{stucture map} of $(x,h)$.
    A \textit{morphism of $T$-algebras} $f \colon (x,h) \to (x',h')$ is an arrow
    $f \colon x \to x'$ such that
    \[
        \begin{tikzcd}
            Tx \arrow[r,"Tf"] \arrow[d,"h"] 
            & Tx' \arrow[d,"h'"] \\
            x \arrow[r,"f"]
            & x'
        \end{tikzcd}
    \]
    commutes.
    The set of all $T$-algebras together with their morphisms form a category,
    which is called the \textit{Eilenberg-Moore-Category} and denoted by $\cat{X}^T$.
\end{definition}
\todo{Proof that this is indeed a category?}
\begin{theorem}[Every monad is defined by its T-algebras]
    Let $(T,\eta,\mu)$ be a monad in a category $\cat{X}$.
    Then there is an adjunction $F^T \adj G^T$, where $F^T$ and $G^T$ are functors
    \begin{tikzcd}
        \cat{X^T} \ar[r,"G^T",shift left = .60ex]
          & 
        \cat{X} \ar[l,"F^T",shift left = .60ex]
    \end{tikzcd}
    such that the monad induced by this adjunction is $(T,\eta,\mu)$.
\end{theorem}
\begin{beweis}
Define $F^T \colon \cat{X} \to \cat{X}^T$ by
\[
    \begin{tikzcd}
        x \arrow[r,mapsto] \arrow[d,"f"]
          & (Tx,\mu_x) \arrow[d,"Tf"] \\
        x' \arrow[r,mapsto]
          & (Tx',\mu_{x'})
    \end{tikzcd}
\]
$(Tx,\mu_x)$ is indeed a $T$-algebra, since $\mu_x$ is an arrow $T^2x \to Tx$
and the diagrams 
\begin{figure}[H]
\centering
\begin{subfigure}{0.4\textwidth}
\centering
%\phantomsection\label{name1}
\begin{tikzcd}
    T^3x \arrow[r,"T(\mu_x)"] \arrow[d,"\mu_{Tx}"] 
      & T^2x \arrow[d,"\mu_x"] \\
    T^2x \arrow[r,"\mu_x"]
      & Tx
\end{tikzcd}
%\caption*{(cap1)}
\end{subfigure}
\hspace{2em}
\begin{subfigure}{0.4\textwidth}
\centering
%\phantomsection\label{name2}
\begin{tikzcd}
    Tx \ar[r,"\eta_{Tx}"] \ar[rd,"id_{Tx}"']
    & T^x \ar[d,"\mu_x"] \\
    & Tx 
\end{tikzcd}
%\caption*{(cap2)}
\end{subfigure}
\end{figure}
are just the commuting diagrams for the~\refassociativity
respectively left~\refunitality axioms from the definition of a Monad.

$Tf \colon (Tx,\mu_x) \to (Tx',\mu_{x'})$ is indeed a morphism of $T$-algebras,
since the commutativity of 
\[
  \begin{tikzcd}
      T^2x \arrow[r,"T^2(f)"] \arrow[d,"\mu_x"] 
        & T^2x' \arrow[d,"\mu_{x'}"] \\
      Tx \arrow[r,"T(f)"]
        & Tx'
  \end{tikzcd}  
\]
is given by naturality of $\mu$. The functoriality of $F^T$ follows from the functoriality
of $T$.

Define $G^T \colon \cat{X^T} \to \cat{X}$ by
\[
    \begin{tikzcd}
        (x,h) \arrow[r,mapsto] \arrow[d,"f"] 
          & x \arrow[d,"f"] \\
        (x',h') \arrow[r,mapsto]
          & x'
    \end{tikzcd}   
\]
so $G$ is just the forgetful functor.
\begin{claim*}
    $G^T \circ F^T = T$ and $F^TG^T(x,h) = (Tx,\mu_x$).
\end{claim*}
\begin{proof}[Proof of claim]
    Let $x \in \cat{X}.$ Then $G^T(F^T(x)) = G^T(Tx,\mu_x) = Tx.$
    Now let $f \colon x \to y$. Then $G^T(F^T(f))=G^t(Tf)=Tf.$
    Finally, $F^TG^T(x,h) = F^T(x) = (Tx,\mu_x).$
\end{proof}
So we can set 
\[
  \eta^T := \eta \colon \id_{\cat{X}} \Rightarrow G^TF^T 
\]
as the unit and we can define $\eps^T \colon F^TG^T \to \id_{\cat{X}^T}$ by
\[
    \eps^T_{(x,h)} := h \colon (Tx,\mu_x) \to (x,h).
\]
$h$ is a morphism of $T$-algebras because $(x,h)$ is a $T$-algebra, since both statements
mean that the diagram 
\[
    \begin{tikzcd}
        T^2x \arrow[r,"Th"] \arrow[d,"\mu"] 
          & Tx \arrow[d,"h"] \\
        T \arrow[r,"h"]
          & x
    \end{tikzcd}
\]
commutes. $\eps^T$ is natural, because if $f \colon (x,h) \to (x',h')$
is a morphism of $T$-algebras, naturality means that the diagram
\[
    \begin{tikzcd}
        Tx \arrow[r,"Tf"] \arrow[d,"h"]
          & Tx' \arrow[d,"h'"] \\
        x \arrow[r,"f"]
          & x'
    \end{tikzcd}
\]
but this is exactly the definition of $f$ being a morphism of $T$-algebras.
\todo{triangle identies and induces T}
\end{beweis}
\begin{theorem}[Comparison of adjunctions with algebras]

\end{theorem}
\section{The Kleisli category of a Monad}
There is another way to induce a Monad by an adjunction:
\begin{definition}[Kleisli category]
    Let $\cat{X}$ be a category, $T = (T,\eta, \mu)$ be a monad in $\cat{X}$.
    The \textit{Kleisli category $\cat{X}_T$} is defined by
    \begin{itemize}
        \item objects the same as in $\cat{X}$, but we relabel $x$ to $x_T$ for all $x \in \cat{X}$.
        \item for $x_T, y_T \in \cat{X}_T$, $f\colon x \to Ty$ is a morphism which we
        denote by $f^b \colon x_T \to y_T$.
        \item composition will be denoted by $\bullet$ for distinction and is defined by
        \[
            g^b \bullet f^b := (\mu_z \circ Tg \circ f)^b \colon x_T \to z_T
        \]
        for $f^b \colon x_T \to y_T$, $g^b \colon y_T \to z_T$. 
        This is indeed again a morphism:
        \begin{tikzcd}
            x \arrow[r,"f"]
              & Ty \arrow[r,"Tg"]
                & T^2z \arrow[r,"\mu_z"]
                  & Tz
        \end{tikzcd}
    \end{itemize}
\begin{claim*}
    This defines a category.
\end{claim*}
\begin{proof}[Proof of claim]
\underline{associativity}: Let
\begin{tikzcd}
    x_T \arrow[r,"f^b"]
      & y_T \arrow[r,"g^b"]
        & z_T \arrow[r,"h^b"]
          & w_T
\end{tikzcd}
be objects and morphisms in the Kleisli category.
\begin{align*}
    (h^b \bullet g^b) \bullet f^b &= (\mu_w \circ Th \circ g)^b \bullet f^b \\
    &= (\mu_w \circ T(\mu_w \circ Th \circ g) \circ f)^b \\
    &= (\mu_w \circ T\mu_w \circ T^2h \circ Tg \circ f)^b.
\end{align*}
Now the~\refassociativity axiom for the Monad T states that
\[
    \begin{tikzcd}
        T(T(Tw)) \ar[r,"T(\mu_w)"] \ar[d,"\mu_{Tw}"'] 
        & T(Tw) \ar[d,"\mu_w"] \\
        T(Tw) \ar[r,"\mu_w"]
        & Tw 
    \end{tikzcd}
\]
commutes, hence 
\[
    (\mu_w \circ T\mu_w \circ T^2h \circ Tg \circ f)^b
    = (\mu_w \circ \mu_{Tw} \circ T^2h \circ Tg \circ f)^b
\]
By naturality of $\mu$, the diagram
\[
    \begin{tikzcd}
        T^2z \arrow[r] \arrow[d] 
          & T^3w \arrow[d] \\
        Tz \arrow[r]
          & T^2w
    \end{tikzcd}
\]
commutes, so it follows that
\begin{align*}
    (\mu_w \circ \mu_{Tw} \circ T^2h \circ Tg \circ f)^b
    &= (\mu_w \circ Th \circ \mu_z \circ Tg \circ f)^b \\
    &= h^b \bullet (g^b \bullet f^b) \\
\end{align*}
\underline{identity axiom}: Let $f^b \colon x_T \to y_T$ be a morphism.
\begin{align*}
    f^b \bullet (\eta_x)^b = (\mu_x \circ Tf \circ \eta_x)^b
    = (\mu_x \circ \eta_{Ty} \circ f)^b
    = (\id_{Ty} \circ f)^b = f^b
\end{align*}
where the second equality follows from the naturality of $\eta$ and 
the third equality is due to the left~\refunitality law for T.
\begin{align*}
    (\eta_y)^b \bullet f^b = (\mu_y \circ T\eta_y \circ f)^b
    =(id_{Ty}\circ f)^b = f^b
\end{align*}
where the second equality is due to the right~\refunitality law for T.
This proves that for $x_T \in \cat{X}_T$ we have $\id_{x_T} = (\eta_x)^b
\in \Hom[\cat{X}_T](x_T,x_T)$
\end{proof}
\begin{theorem}
    There is an adjoint situation $F_T \adj G_T \colon$ 
    \begin{tikzcd}
        \cat{X_T} \ar[r,,shift left = .60ex]
          & 
        \cat{X} \ar[l,,shift left = .60ex]
    \end{tikzcd}
    such that $T$ is the induced monad by this adjunction.
\end{theorem}
\end{definition}
\chapter{Witt vectors}
\section*{Construction of the witt vectors}

Recall that for every prime number $p$, we have the \textit{p-adic valuation map}:
\begin{definition}[p-adic valuation]
   $v_p \colon \mathbb{Z} \to \mathbb{N} \cup\{\infty\}$
    is defined by 
    \[
        v_p(n)=
        \begin{cases}
        \mathrm{max}\{k \in \mathbb{N} : p^k \mid n\} & \text{if } n \neq 0\\
        \infty & \text{if } n=0
        \end{cases} 
    \]

\end{definition}

\begin{definition}[truncation set]
    Let $\N$ be the set of positive integers and let $S\subseteq
    \N$ be a subset with the property that $\forall n\in \N:$
    if $d$ is a divisor of $n$, then $d\in S$.
    We then say that S is a \textit{truncation set}.
\end{definition}
As a set, we define the \textit{big Witt ring} $\W_S(A)$ to be $A^S$,
we will give it a unique ring structure, such that the \textit{ghost map}
is a ring homomorphism.

\begin{definition}[ghost map]
    We define $w \colon \W_S(A) \to A^S$
    by $(a_n)_{n \in S} \mapsto (w_n)_{n \in S}$ where 
    \[
        w_n = \sum_{d \mid n} d a_d^{n/d}
    \]
\end{definition}
The core of the construction is contained in the following Lemma:
\begin{lemma}[Dwork]\label{lem: dwork}
    Suppose that for every prime number
$p$ there exists a ring homomorphism $\phi_p \colon A \to A$ with
the property that $\phi_p(a) \equiv a^p$ modulo $pA$. Then for every
sequence $x = (x_n)_{n \in S}$, the following 
are equivalent:
\begin{enumerate}[(i)]
\item The sequence $x$ is in the image of the ghost map
$w \colon \mathbb{W}_S(A) \to A^S.$
\item For every prime number $p$ and every $n \in S$
with $v_p(n) \geqslant 1$,
$$x_n \equiv \phi_p(x_{n/p}) \hskip8mm \text{modulo $p^{v_p(n)}A$.}$$
\end{enumerate}    
\end{lemma}
\begin{bigproof}
    $\hin$ Suppose $x$ is in the image of the ghost map, that means there is a sequence 
    $a = (a_n)_{n \in S}$ such that $x_n = w_n(a)$ for all $n \in S$. 
    We calculate:
    \[
        \phi(x_{n/p}) = \phi(w_{n/p}(a)) = \phi(\sum_{d \mid n/p} d a_d^{n/pd}) =
        \sum_{d \mid n/p} d \cdot \phi(a_d^{n/pd}) 
    \] 
    since $\phi$ is a ring homomorphism and $d \in \N$.
    \begin{claim}
        \[
        \sum_{d \mid n/p} d \cdot \phi(a_d^{n/pd}) \equiv 
        \sum_{d \mid n/p} d \cdot a_d^{n/d} \text{\hskip8mm mod} \  p^{v_p(n)}A.
        \]
    \end{claim}
    \begin{smallproof}
        \todo{}
    \end{smallproof}
    
    \begin{claim}
        \[
        \sum_{d \mid n/p} d \cdot a_d^{n/d} \equiv 
        \sum_{d \mid n} d \cdot a_d^{n/d} \text{\hskip8mm mod} \ p^{v_p(n)}A
        \]
    \end{claim} 
    \begin{smallproof}
        \todo{}
    \end{smallproof}
    so we get
    \[
        \phi(x_{n/p}) \equiv \sum_{d \mid n} d \cdot a_d^{n/d} = w_n(a) = x_n \hskip8mm mod \ p^{v_p(n)}A.
        \]
    
    $\rueck$ Let $(x_n)_{n \in S}$ be a sequence such that 
    $x_n \equiv \phi_p(x_{n/p}) \hskip8mm mod \ p^{v_p(n)}A \ \forall p\ $prime$, n\in S, v_p(n) \geqslant 1.$
    Define $(a_n)_{n \in S}$ with $w_n(a) = x_n$ as follows:
    \[a_1 := x_1\]
    and if $a_d$ has been chosen for all $d \mid n$ such that $w_d(a) = x_d$ we see that
    \begin{align*}
            x_n &\equiv \phi_p(x_{n/p}) \hskip8mm mod \ p^{v_p(n)}A \\
                &= \phi_p(\sum_{d \mid n/p} d \cdot a_d^{n/pd}) \\
                &= \sum_{d \mid n/p} d \cdot \phi(a_d^{n/pd})
    \end{align*}
 \todo{finish proof}
\end{bigproof}
We will often need the following
\begin{lemma} \label{lem: injective ghost map}
    if $A$ is a torsion-free ring, the ghost map is injective.
\end{lemma}
Now we can finish the construction of the Witt vectors:
\begin{theorem} \label{thm: existence of witt vectors}
    There exists a unique ring structure such that the ghost map 
    \[
      w:\W_S(A) \to A^s  
    \]
    is a natural transformation of functors from rings to rings.
\end{theorem}
\begin{bigproof}
    
\end{bigproof}
\begin{cor} \label{cor: ghost components are nat trafos}
    $w_n \colon \W_S(A) \to A$ is a natural ring homomorphism for all $n \in S$.
\end{cor}
\begin{proposition} \label{prop: W is a functor}
    $\W_S$ is a functor $\cat{CRing} \to \cat{CRing}$.
\end{proposition}
\section*{The Verschiebung, Frobenius and Teichmüller maps}
We have various operations on witt vectors that are of interest.
\begin{definition}[Restriction map]
    If $T \subseteq S$ are two truncation sets, the \textit{restriction from S to T}
    \[
      R_T^S \colon \W_S(A) \to \W_T(A)  
    \]
    is a natural ring homomorphism.
\end{definition}
If $S\subseteq \N$ is a truncation set, $n \in \N$, then
\[
   S/n := \{d \in \N \mid nd \in S\}
\]
is again a truncation set.
\begin{definition}[Verschiebung] \
    Define 
    \[
        V_n \colon \W_{S/n} \to \W_S(A);\  
        V_n((a_d)_{d \in S/n})_m := 
        \begin{cases}
            a_d, &\quad \text{if $m=n \cdot d$} \\
            0,  &\quad \text{else}
        \end{cases}
    \]
    which is called the \textit{n-th Verschiebung map}. Furthermore define
    \[
        \widetilde{V_n} \colon A^{S/n} \to A^S;\ 
        \widetilde{V_n}((x_d)_{d \in S/n})_m := 
        \begin{cases}
            n \cdot x_d, &\quad \text{if $m=n \cdot d$} \\
            0,  &\quad \text{else}
        \end{cases}
    \]
\end{definition}
\begin{lemma} \label{lem: verschiebung is additive}
    The Verschiebung map $V_n$ is additive.
\end{lemma}
\begin{bigproof}
    \begin{claim*}
        \begin{tikzcd}
            \W_{S/n}(A) \arrow[r,"w"] \arrow[d,"V_n"]
            & A^{S/n} \arrow[d,"\widetilde{V_n}"]\\
            \W_S(A) \arrow[r,"w"]
            & A^S
        \end{tikzcd}
        commutes.
    \end{claim*}
    \begin{smallproof}
        
    \end{smallproof}
\end{bigproof}
Define $\widetilde{F_n} \colon A^S \to A^{S/n} \ $
by $\widetilde{F_n}((x_m)_{m \in S})_d = x_{nd}$.
\begin{lemma}[Frobenius homomorphism] \label{lem: frobenius}
    There exists a unique natural ring homomorphism
    \[
      F_n \colon \W_S(A) \to \W_{S/n}(A)  
    \]
    such that the diagram 
    \[
        \begin{tikzcd}
            \W_S(A) \arrow[r,"w"] \arrow[d,"F_n"] 
              & A^S \arrow[d,"\widetilde{F_n}"] \\
            \W_{S/n}(A) \arrow[r,"w"]
              & A^{S/n}
        \end{tikzcd}        
    \]
    commutes.
\end{lemma}
\todo{remark und definition haben andere font}

    We call $F_n$ the \textit{nth Frobenius homomorphism}.
    The commutativity of the diagram above is equivalent to
    commutativity of the following diagram for every $d \in S/n$:
    \[
        \begin{tikzcd}
            \W_S(A) \arrow[d,"F_n"] \arrow[dr,"w_{nd}"] \\
            \W_{S/n}(A) \arrow[r,"w_d"'] 
            & A
        \end{tikzcd}
    \]

\begin{proof}[Proof of Lemma~\ref*{lem: frobenius}]
    easy
\end{proof}
\begin{lemma} \label{lem: F_n after F_m is F_{nm}}
    Let $n,m \in \N$.
    Then 
    \[F_n \circ F_m = F_{nm}.\]
\end{lemma}
\begin{beweis}
    \todo{}
\end{beweis}
\begin{definition}[teichmüller representative]
    The \textit{teichmüller representative} is the map
    \[
      \tau \colon A \to \W_S(A)  
    \]
    defined by
    \[
      (\tau(a))_m =   
      \begin{cases}
        a, & \text{if } m = 1\\
        0, & \text{else}
        \end{cases}
    \]
\end{definition}
\begin{lemma} \label{lem: teichmüller is multiplicative}
    The teichmüller map is multiplicative.
\end{lemma}
\begin{beweis}
    The map $\widetilde{\tau} \colon A \to A^S$; $(\widetilde{\tau})_n = a^n$
    is multiplicative and there is a commutative diagram
    \[
        \begin{tikzcd}
              & A \arrow[rd,"\widetilde{\tau}"] \\ %so kann man btw pfeile umdrehen (leftarrow)
            \W_S(A) \arrow[ru, leftarrow,"\tau"] \arrow[rr,"w"]
                && A^S.
        \end{tikzcd}
    \]
    Indeed, $w_n(\tau(a)) = w_n((a,0,0,\dots)) = a^n$
    by definition of $w_n$.

\end{beweis}
\section*{The comonad structure of witt vectors}
We will need the following lemma:
\begin{lemma}\label{lem: non-zero divisor}
    Let $m \in \Z$. If $m$ is a non-zero divisor in A, then it is a
    non-zero divisor in $\W_S(A)$ as well.
\end{lemma}
\begin{beweis}
    \[
    \begin{tikzcd}
        0 \arrow[r]
          & A \arrow[r,"V_n"]
            & \W_S(A) \arrow[r,"R_T^S"]
              & W_T(A) \arrow[r]
                & 0
    \end{tikzcd}
    \]
    which we can extend to the following commutative diagram:
    \[
    \begin{tikzcd}
        0 \arrow[r] 
        & A \arrow[r] \arrow[d,"\cdot m"] 
          & \W_S(A) \arrow[r] \arrow[d,"\cdot m"]
            & \W_T(A) \arrow[d,"\cdot m"] \arrow[r]
                & 0 \\
        0 \arrow[r]
           & A \arrow[r]
            & \W_S(A) \arrow[r]
              & \W_T(A) \arrow[r]
                & 0 
    \end{tikzcd}
    \]
    \todo{finish}
\end{beweis}
\begin{cor} 
    \label{cor: A torsion-free implies W(A) torsion-free}
    If $A$ is torsion-free, then $\W_S(A)$ is torsion-free as well.
\end{cor}
\begin{definition}
    $\W(A) := \W_{\N}(A)$
\end{definition}
For the construction of a natural transformation $\W(A) \to \W(\W(A))$
we want to use Lemma ~\ref{lem: dwork} again. Hence we first show:
\begin{lemma} \label{lem: frobenius lifts frobenius}
    Let $p$ be a prime number, let $A$ be any ring.
    Then the ring homomorphism $F_p \colon \W(A) \to \W(A)$  
    satisfies $F_p(a) \equiv a^p \ mod \ pA.$
\end{lemma}
\begin{proposition} \label{prop: existence of diagonal}
    There exists a unique natural transformation
    \[
      \Delta \colon \W(A) \to \W(\W(A))  
    \]
    such that $w_n(\Delta(a))=F_n(A)$ for all $a \in A, n \in \N$.
\end{proposition}

Recall that by ~\ref{cor: ghost components are nat trafos},
$w_1 \colon \W(A) \to A$; $(a_n)_{n \in \N} \mapsto a_1$
is a natural transformation $\W \Rightarrow \id_{\cat{CRing}}$.
\begin{theorem} \label{thm: comonad structure}
    The functor $\W(\_) \colon \cat{CRing} \to \cat{CRing}$ together with the
    natural transformations $\Delta \colon \W \Rightarrow \W^2,$ $w_1 \colon 
    \W \Rightarrow \id_{\cat{CRing}}$ form a comonad $(\W,w_1,\Delta)$.
\end{theorem}
\begin{bigproof}
    \begin{claim*}
        
        \begin{tikzcd}
            \W(A) \arrow[r,"\Delta_A"] \arrow[d,"\Delta_A"] \arrow[dr,phantom,"\#"]
             & \W(\W(A)) \arrow[d,"\W(\Delta_A)"] \\
            \W(\W(A)) \arrow[r,"\Delta_{\W(A)}"]
              & \W(\W(\W(A)))
        \end{tikzcd}
        commutes.
        
    \end{claim*}
    \begin{smallproof}
        evaluating the ghost coordinates leads to:
        \[
            \begin{tikzcd}
                \W(A) \arrow[r,"\Delta_A",] \arrow[d,"\Delta_A"] \arrow[rr,bend left,"F_A",dotted]
                 & \W(\W(A)) \arrow[d,"\W(\Delta_A)"] \arrow[r,"w",dotted] 
                 & \W(A)^{\N} \arrow[d,"\Delta_A^{\N}",dotted]\\
                \W(\W(A)) \arrow[r,"\Delta_{\W(A)}"] \arrow[rr,bend right,"F_{\W_A}",dotted]
                  & \W(\W(\W(A))) \arrow[r,"w",blue]
                  & \W(\W(A))^{\N}
            \end{tikzcd}  
        \]
        which by ~\ref{prop: existence of diagonal} simplifies to
        \[
            \begin{tikzcd}
                \W(A) \arrow[r,"F_A"] \arrow[d,"\Delta_A"] 
                 & \W(A)^{\N} \arrow[d,"\Delta_A^{\N}"] \\
                \W(\W(A)) \arrow[r,"F_{\W(A)}"]
                  & \W(\W(A))^{\N}
            \end{tikzcd}
        \]
        now it suffices to show for an arbitrary n that the following diagram commutes:
        \[
            \begin{tikzcd}
                \W(A) \arrow[r,"F_{n_A}"] \arrow[d,"\Delta_A"] 
                 & \W(A) \arrow[d,"\Delta_A"] \\
                \W(\W(A)) \arrow[r,"F_{n_{\W(A)}}"]
                  & \W(\W(A))
            \end{tikzcd}
        \]
        evaluating the ghost coordinates again, keeping in mind that by 
        ~\ref{cor: A torsion-free implies W(A) torsion-free}
        and ~\ref{lem: injective ghost map}, 
        $w \colon \W(\W(A)) \to \W(A)^{\N}$ is injective as well, we get
        \[
            \begin{tikzcd}
                \W(A) \arrow[r,"F_{n_A}"] \arrow[d,"\Delta_A"] 
                 & \W(A) \arrow[d,"\Delta_A"] \arrow[dd,bend left = 60,"F_A",dotted]\\
                \W(\W(A)) \arrow[r,"F_{n_{\W(A)}}"] \arrow[d,"w",dotted]
                  & \W(\W(A)) \arrow[d,"w",blue] \\
                \W(A)^{\N} \arrow[r,"\widetilde{F_n}_{\W(A)}",dotted]
                & \W(A)^{\N}
            \end{tikzcd}
        \]
        using the fact that 
        \begin{tikzcd}
            \W(\W(A)) \arrow[d,"w",dotted] \arrow[rd,"w_{nm}",dotted]\\
            \W(A)^{\N} \arrow[r,"\widetilde{F_n}_{\W(A)}",dotted]
            & \W(A)^{\N}
        \end{tikzcd}
        commutes, we can simplify the situation to
        \[
            \begin{tikzcd}
            \W(A) \arrow[r,"F_n"] \arrow[d,"\Delta_A"] \arrow[rd,"F_{nm}",dotted]
                 & \W(A) \arrow[d,"F_m"] \\
                \W(\W(A)) \arrow[r,"w_{nm}"] 
                  & \W(A) \\
            \end{tikzcd}
        \]
        which can again be simplified to
        \[
            \begin{tikzcd}
                \W(A) \arrow[r,"F_n"] \arrow[rd,"F_{nm}"']
                & \W(A) \arrow[d,"F_m"]\\
                & \W(A)
            \end{tikzcd}
        \]
        now this commutes by ???, hence we are finished.
    \end{smallproof}
    \begin{claim*}
        \begin{tikzcd}
            \W(A) \arrow[d,"\Delta_A"'] \arrow[rd,"\id_{\W(A)}"]\\
            \W(\W(A)) \arrow[r,"\W(w_1)"']
            & \W(A) 
        \end{tikzcd}
        commutes.
    \end{claim*}
    \begin{smallproof}
        evaluate the ghost coordinates:
        \[
            \begin{tikzcd}
                \W(A) \arrow[d,"\Delta_A"'] \arrow[rd,"\id_{\W(A)}"] 
                \arrow[dd,bend right = 60,"F"',dotted]\\
                \W(\W(A)) \arrow[r,"\W(w_1)"'] \arrow[d,"w",dotted]
                & \W(A) \arrow[d,"w",blue] \\
                \W(A)^{\N} \arrow[r,"w_1^{\N}",dotted]
                & A^{\N}
            \end{tikzcd}
        \]
    we can then simplify to
    \[
        \begin{tikzcd}
            \W(A) \arrow[d,"F"'] \arrow[rd,"w"] \\
            \W(A)^{\N} \arrow[r,"w_1^{\N}"'] 
            & A^{\N}
        \end{tikzcd}
    \]
    now it suffices to show for all $n$ that 
    \[
      \begin{tikzcd}
        \W(A) \arrow[d,"F_n"'] \arrow[dr,"w_n"]\\
        \W(A) \arrow[r,"w_1"'] 
        & A
      \end{tikzcd}
    \]
    commutes, which is true by ??? ($\eps = w_1$).

    \end{smallproof}
    \begin{claim*}
        \begin{tikzcd}
            & \W(A) \arrow[d,"\Delta_A"] \arrow[ld,"\id_{\W(A)}"'] \\
            \W(\W(A))  & \W(A) \arrow[l,"\eps_{\W(A)}"]
        \end{tikzcd}
        commutes.
    \end{claim*}
    \begin{smallproof} 
        Let $a \in \W(A)$. \\
        $\eps(\Delta_A(a)) = w_1(\Delta_A(a)) = F_1(a) = a$,
        since $F_1 = \id_{\W(A)}$.
    \end{smallproof}
    This concludes the proof.
\end{bigproof}
\section*{The Teichmüller map induces a morphism of comonads}
We now consider another example of a comonad; the \textit{free monoid comonad}.
\todo{maybe ins Kapitel Comonaden als Beispiel schieben}
\begin{definition}[monoid ring]
    Let $R$ be a ring and let $G$ be a monoid.
    The \textit{monoid ring} of $G$ over $R$, denoted $R[G]$ or $RG$
    is the set of formal finite sums $\sum_{g \in G}r_g \cdot g$
    with addition and multiplication defined by:
    \begin{align*}
        \sum_{g \in G}r_g \cdot g + \sum_{g \in G}s_g \cdot g
        & := \sum_{g \in G}(r_g + s_g)\cdot g \\
        \sum_{g \in G}r_g \cdot g \cdot \sum_{g \in G}s_g \cdot g
        & := \sum_{g \in G}(\sum_{k \cdot l = g} r_k \cdot s_l)\cdot g 
    \end{align*}
\end{definition}
\begin{example}
    $R = \R, G = \{x^n \mid n \in \N\} \implies RG = \R[X]$
\end{example}
\begin{remark} \label{rem: universal property of monoid ring}
$R[G]$ together with the ring homomorphism $\alpha \colon R \to R[G]$;
$r \mapsto r \cdot 1$ and the monoid homomorphism $\beta \colon 
G \to R[G]$; $g \mapsto 1 \cdot g$ 
enjoys the following universal property:
\[
  \alpha(r) \cdot \beta(g) = \beta(g) \cdot \alpha(r)
   \quad \forall r \in R, g \in G
\]
and if $(S,\alpha',\beta')$ is another such triple with
$\alpha'(r) \cdot \beta'(g) = \beta'(g) \cdot \alpha'(r)
   \quad \forall r \in R, g \in G$,
there is a unique monoid homomorphism $\gamma \colon R[G] \to S$
such that the following diagram commutes:
\[
    \begin{tikzcd}
        & S \\
        R \arrow[r,"\alpha"'] \arrow[ur,"\alpha'"] 
        & R[G] \arrow[u,"\gamma",dotted] 
        & G \arrow[l,"\beta"] \arrow[ul,"\beta'"']
    \end{tikzcd}
\]
Here, $\gamma$ is defined by 
$\sum_{g \in G}r_g \cdot g \mapsto \sum_{g \in G}\alpha'(r_g) \cdot \beta'(g)$.
\end{remark}
\begin{example}
    Let $S$ be a ring, $G$ be a monoid.
    Since there is a unique ring homomorphism $\Z \to S$, 
    each monoid homomorphism $G \to S$ induces a unique ring homomorphism
    $\Z G \to S$ such that the following commutes:
    \[
      \begin{tikzcd}
        G \arrow[r] \arrow[rd] 
        &S \\
        & \Z G \arrow[u]
      \end{tikzcd}
    \]
    Now if $H$ is another monoid and $f \colon G \to H$ a monoid morphism,
    $G \xrightarrow{f} H \to \Z H$ is a monoid homomorphism,
    hence it extends uniquely to $f \colon \Z G \to \Z H$,
    $\sum_{g \in G}r_g \cdot g \mapsto \sum_{g \in G}r_g \cdot f(g)$.\\ 
    In this way, the free monoid ring construction over $\Z$ is functorial.
\end{example}

Let $G \colon \cat{CRing} \to \cat{CMon}$, $(R,+,\cdot) \mapsto (R,\cdot)$
be the forgetful functor and
let $F \colon \cat{CMon} \to \cat{CRing}$ be the \textit{free monoid ring functor},
$G \mapsto \Z G$.
\begin{proposition} \label{prop: adjunction monoid ring}
    There is an adjoint situation \begin{tikzcd}
        \cat{CMon}
            \arrow[r, bend left = 25, "F"{name=D}]
            \arrow[r, leftarrow, bend right = 25, swap, "G"{name=C}]
              \arrow[d, from=D, to=C, phantom, "{\bot}"]
          & \cat{CRing}
    \end{tikzcd}
\end{proposition}
Now consider the \textit{teichmüller map} $\tau \colon A \to \W(A); 
a \mapsto (a,0,0,0,\dots)$.
$\tau$ is multiplicative and preserves the unit, hence it extends uniquely to 
a ring homomorphism
\[
\tau \colon \Z A \to \W(A)
\]
\begin{theorem} \label{thm: morphism of comonads}
    $\tau \colon \Z A \to \W(A)$ is a morphism of comonads.
\end{theorem}



% compiliert sonst zu lange
\printbibliography[title=Literature]
\end{document}
