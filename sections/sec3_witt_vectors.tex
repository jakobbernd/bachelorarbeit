\chapter{Witt vectors}
\section{Construction of the witt vectors}
\begin{definition}[truncation set]
    Let $\N$ be the set of positive integers and let $S\subseteq
    \N$ be a subset with the property that $\forall n\in \N:$
    if $d$ is a divisor of $n$, then $d\in S$.
    We then say that S is a \textit{truncation set}.
\end{definition}
As a set, we define the \textit{big Witt ring} $\W_S(A)$ to be $A^S$,
we will give it a unique ring structure, such that the \textit{ghost map}
is a ring homomorphism.
\begin{definition}[ghost map]
    We define $w \colon \W_S(A) \to A^S$
    by $(a_n)_{n \in S} \mapsto (w_n)_{n \in S}$ where 
    \[
        w_n = \sum_{d \mid n} d a_d^{n/d}
    \]
\end{definition}
\begin{lemma}[Dwork]
    Suppose that for every prime number
$p$ there exists a ring homomorphism $\phi_p \colon A \to A$ with
the property that $\phi_p(a) \equiv a^p$ modulo $pA$. Then for every
sequence $x = (x_n)_{n \in S}$, the following 
are equivalent:
\begin{enumerate}[(i)]
\item The sequence $x$ is in the image of the ghost map
$w \colon \mathbb{W}_S(A) \to A^S.$
\item For every prime number $p$ and every $n \in S$
with $v_p(n) \geqslant 1$,
$$x_n \equiv \phi_p(x_{n/p}) \hskip8mm \text{modulo $p^{v_p(n)}A$.}$$
\end{enumerate}    
\end{lemma}
\begin{bigproof}
    $\hin$ Suppose $x$ is in the image of the ghost map, that means there is a sequence 
    $a = (a_n)_{n \in S}$ such that $x_n = w_n(a)$ for all $n \in S$. 
    We calculate:
    \[
        \phi(x_{n/p}) = \phi(w_{n/p}(a)) = \phi(\sum_{d \mid n/p} d a_d^{n/pd}) =
        \sum_{d \mid n/p} d \cdot \phi(a_d^{n/pd}) 
    \] 
    since $\phi$ is a ring homomorphism and $d \in \N$.
    \begin{claim}
        $\sum_{d \mid n/p} d \cdot \phi(a_d^{n/pd}) \equiv 
        \sum_{d \mid n/p} d \cdot a_d^{n/d}$ \hskip8mm mod $p^{v_p(n)}A$.
    \end{claim}
    \begin{smallproof}
        \todo{}
    \end{smallproof}
    
    \begin{claim}
        $\sum_{d \mid n/p} d \cdot a_d^{n/d} \equiv 
        \sum_{d \mid n} d \cdot a_d^{n/d} \hskip8mm mod \ p^{v_p(n)}A$
    \end{claim} 
    \begin{smallproof}
        \todo{}
    \end{smallproof}
    so we get
    \[
        \phi(x_{n/p}) \equiv \sum_{d \mid n} d \cdot a_d^{n/d} = w_n(a) = x_n \hskip8mm mod \ p^{v_p(n)}A.
        \]
    
    $\rueck$ Let $(x_n)_{n \in S}$ be a sequence such that 
    $x_n \equiv \phi_p(x_{n/p}) \hskip8mm mod \ p^{v_p(n)}A \ \forall p\ $prime$, n\in S, v_p(n) \geqslant 1.$
    Define $(a_n)_{n \in S}$ with $w_n(a) = x_n$ as follows:
    \[a_1 := x_1\]
    and if $a_d$ has been chosen for all $d \mid n$ such that $w_d(a) = x_d$ we see that
    \begin{align*}
            x_n &\equiv \phi_p(x_{n/p}) \hskip8mm mod \ p^{v_p(n)}A \\
                &= \phi_p(\sum_{d \mid n/p} d \cdot a_d^{n/pd}) \\
                &= \sum_{d \mid n/p} d \cdot \phi(a_d^{n/pd})
    \end{align*}
 \todo{finish proof}
\end{bigproof}
We will often need the following
\begin{lemma}
    if $A$ is a torsion-free ring, the ghost map is injective.
\end{lemma}
Now we can finish the construction of the Witt vectors:
\begin{theorem}
    There exists a unique ring structure such that the ghost map 
    \[
      w:\W_S(A) \to A^s  
    \]
    is a natural transformation of functors from rings to rings.
\end{theorem}
\begin{bigproof}
    
\end{bigproof}
\begin{cor}
    $w_n \colon \W_S(A) \to A$ is a natural transformation for all $n \in S$.
\end{cor}
\begin{proposition}
    $\W_S$ is a functor $\cat{CRing} \to \cat{CRing}$.
\end{proposition}
\section{The Verschiebung, Frobenius and Teichmüller maps}
If $S\subseteq \N$ is a truncation set, then
\[
   S/n := \{d \in \N \mid nd \in S\}
\]
is again a truncation set.
\begin{definition}[Verschiebung]
    Define 
    \[
        V_n \colon \W_{S/n} \to \W_S(A);\  
        V_n((a_d)_{d \in S/n})_m := 
        \begin{cases}
            a_d, &\quad \text{if $m=n \cdot d$} \\
            0,  &\quad \text{else}
        \end{cases}
    \]
    which is called the \textit{n-th Verschiebung map}. Furthermore define
    \[
        \widetilde{V_n} \colon A^{S/n} \to A^S;\ 
        \widetilde{V_n}((x_d)_{d \in S/n})_m := 
        \begin{cases}
            n \cdot x_d, &\quad \text{if $m=n \cdot d$} \\
            0,  &\quad \text{else}
        \end{cases}
    \]
\end{definition}
\begin{lemma}
    The Verschiebung map $V_n$ is additive.
\end{lemma}
\section{The comonad structure of witt vectors}
We will need the following lemma:
\begin{lemma}
    Let $m \in \Z$. If $m$ is a non-zero divisor in A, then it is a
    non-zero divisor in $\W_S(A)$ as well.
\end{lemma}
\begin{smallproof}
    \[
    \begin{tikzcd}
        0 \arrow[r]
          & A \arrow[r,"V_n"]
            & \W_S(A) \arrow[r,"R_T^S"]
              & W_T(A) \arrow[r]
                & 0
    \end{tikzcd}
    \]
    which we can extend to the following commutative diagram:
    \[
    \begin{tikzcd}
        0 \arrow[r] 
        & A \arrow[r] \arrow[d,"\cdot m"] 
          & \W_S(A) \arrow[r] \arrow[d,"\cdot m"]
            & \W_T(A) \arrow[d,"\cdot m"] \arrow[r]
                & 0 \\
        0 \arrow[r]
           & A \arrow[r]
            & \W_S(A) \arrow[r]
              & \W_T(A) \arrow[r]
                & 0 
    \end{tikzcd}
    \]
    \todo{finish}
\end{smallproof}
\begin{definition}
    $\W(A) := \W_{\N}(A)$
\end{definition}
For the construction of a natural transformation $\W(A) \to \W(\W(A))$
we want to use Lemma ??? again. Hence we first show:
\begin{lemma}
    Let $p$ be a prime number, let $A$ be any ring.
    Then the ring homomorphism $F_p \colon \W(A) \to \W(A)$  
    satisfies $F_p(a) \equiv a^p \ mod \ pA.$
\end{lemma}
\begin{proposition}
    There exists a unique natural transformation
    \[
      \Delta \colon \W(A) \to \W(\W(A))  
    \]
    such that $w_n(\Delta(a))=F_n(A)$ for all $a \in A, n \in \N$.
\end{proposition}
\begin{theorem}
    The functor $\W(\_) \colon \cat{CRing} \to \cat{CRing}$ together with the
    natural transformations $\Delta \colon \W \Rightarrow \W^2,$ $w_1 \colon 
    \W \Rightarrow \id_{\cat{CRing}}$ form a comonad.
\end{theorem}
\begin{bigproof}
    \begin{claim*}
        
        \begin{tikzcd}
            \W(A) \arrow[r,"\Delta_A"] \arrow[d,"\Delta_A"] \arrow[dr,phantom,"\#"]
             & \W(\W(A)) \arrow[d,"\W(\Delta_A)"] \\
            \W(\W(A)) \arrow[r,"\Delta_{\W(A)}"]
              & \W(\W(\W(A)))
        \end{tikzcd}
        commutes.
        
    \end{claim*}
    \begin{smallproof}
        evaluating the ghost coordinates leads to:
        \[
            \begin{tikzcd}
                \W(A) \arrow[r,"\Delta_A",] \arrow[d,"\Delta_A"] \arrow[rr,bend left,"F_A",dotted]
                 & \W(\W(A)) \arrow[d,"\W(\Delta_A)"] \arrow[r,"w",dotted] 
                 & \W(A)^{\N} \arrow[d,"\Delta_A^{\N}",dotted]\\
                \W(\W(A)) \arrow[r,"\Delta_{\W(A)}"] \arrow[rr,bend right,"F_{\W_A}",dotted]
                  & \W(\W(\W(A))) \arrow[r,"w",blue]
                  & \W(\W(A))^{\N}
            \end{tikzcd}  
        \]
        which simplifies to
        \[
            \begin{tikzcd}
                \W(A) \arrow[r,"F_A"] \arrow[d,"\Delta_A"] 
                 & \W(A)^{\N} \arrow[d,"\Delta_A^{\N}"] \\
                \W(\W(A)) \arrow[r,"F_{\W(A)}"]
                  & \W(\W(A))^{\N}
            \end{tikzcd}
        \]
        now it suffices to show for an arbitrary n that the following diagram commutes:
        \[
            \begin{tikzcd}
                \W(A) \arrow[r,"F_{n_A}"] \arrow[d,"\Delta_A"] 
                 & \W(A) \arrow[d,"\Delta_A"] \\
                \W(\W(A)) \arrow[r,"F_{n_{\W(A)}}"]
                  & \W(\W(A))
            \end{tikzcd}
        \]
        evaluating the ghost coordinates again, keeping in mind that by Lemma 9, 
        $w \colon \W(\W(A)) \to \W(A)^{\N}$ is injective as well, we get
        \[
            \begin{tikzcd}
                \W(A) \arrow[r,"F_{n_A}"] \arrow[d,"\Delta_A"] 
                 & \W(A) \arrow[d,"\Delta_A"] \arrow[dd,bend left = 60,"F_A",dotted]\\
                \W(\W(A)) \arrow[r,"F_{n_{\W(A)}}"] \arrow[d,"w",dotted]
                  & \W(\W(A)) \arrow[d,"w",blue] \\
                \W(A)^{\N} \arrow[r,"\widetilde{F_n}_{\W(A)}",dotted]
                & \W(A)^{\N}
            \end{tikzcd}
        \]
        using the fact that 
        \begin{tikzcd}
            \W(\W(A)) \arrow[d,"w",dotted] \arrow[rd,"w_{nm}",dotted]\\
            \W(A)^{\N} \arrow[r,"\widetilde{F_n}_{\W(A)}",dotted]
            & \W(A)^{\N}
        \end{tikzcd}
        commutes, we can simplify the situation to
        \[
            \begin{tikzcd}
            \W(A) \arrow[r,"F_n"] \arrow[d,"\Delta_A"] \arrow[rd,"F_{nm}",dotted]
                 & \W(A) \arrow[d,"F_m"] \\
                \W(\W(A)) \arrow[r,"w_{nm}"] 
                  & \W(A) \\
            \end{tikzcd}
        \]
        which can again be simplified to
        \[
            \begin{tikzcd}
                \W(A) \arrow[r,"F_n"] \arrow[rd,"F_{nm}"']
                & \W(A) \arrow[d,"F_m"]\\
                & \W(A)
            \end{tikzcd}
        \]
        now this commutes by ???, hence we are finished.
    \end{smallproof}
    \begin{claim*}
        \begin{tikzcd}
            \W(A) \arrow[d,"\Delta_A"'] \arrow[rd,"\id_{\W(A)}"]\\
            \W(\W(A)) \arrow[r,"\W(\eps_A)"']
            & \W(A) 
        \end{tikzcd}
        commutes.
    \end{claim*}
    \begin{smallproof}
        evaluate the ghost coordinates:
        \[
            \begin{tikzcd}
                \W(A) \arrow[d,"\Delta_A"'] \arrow[rd,"\id_{\W(A)}"] 
                \arrow[dd,bend right = 60,"F"',dotted]\\
                \W(\W(A)) \arrow[r,"\W(\eps_A)"'] \arrow[d,"w",dotted]
                & \W(A) \arrow[d,"w",blue] \\
                \W(A)^{\N} \arrow[r,"\eps_A^{\N}",dotted]
                & A^{\N}
            \end{tikzcd}
        \]
    we can then simplify to
    \[
        \begin{tikzcd}
            \W(A) \arrow[d,"F"'] \arrow[rd,"w"] \\
            \W(A)^{\N} \arrow[r,"\eps_A^{\N}"'] 
            & A^{\N}
        \end{tikzcd}
    \]
    now it suffices to show for all $n$ that 
    \[
      \begin{tikzcd}
        \W(A) \arrow[d,"F_n"'] \arrow[dr,"w_n"]\\
        \W(A) \arrow[r,"\eps_A"'] 
        & A
      \end{tikzcd}
    \]
    commutes, which is true by ??? ($\eps = w_1$).

    \end{smallproof}
    \begin{claim*}
        \begin{tikzcd}
            & \W(A) \arrow[d,"\Delta_A"] \arrow[ld,"\id_{\W(A)}"'] \\
            \W(\W(A))  & \W(A) \arrow[l,"\eps_{\W(A)}"]
        \end{tikzcd}
        commutes.
    \end{claim*}
    \begin{smallproof} 
        Let $a \in \W(A)$. \\
        $\eps(\Delta_A(a)) = w_1(\Delta_A(a)) = F_1(a) = a$,
        since $F_1 = \id_{\W(A)}$.
    \end{smallproof}
    This concludes the proof.
\end{bigproof}
\section{The Teichmüller map induces a morphism of comonads}
We now consider another example of a comonad; the \textit{free monoid comonad}.
\begin{definition}[monoid ring]
    Let $R$ be a ring and let $G$ be a monoid.
    The \textit{monoid ring} of $G$ over $R$, denoted $R[G]$ or $RG$
    is the set of formal finite sums $\sum_{g \in G}r_g \cdot g$
    with addition and multiplication defined by:
    \begin{align*}
        \sum_{g \in G}r_g \cdot g + \sum_{g \in G}s_g \cdot g
        & := \sum_{g \in G}(r_g + s_g)\cdot g \\
        \sum_{g \in G}r_g \cdot g \cdot \sum_{g \in G}s_g \cdot g
        & := \sum_{g \in G}(\sum_{k \cdot l = g} r_k \cdot s_l)\cdot g 
    \end{align*}
\end{definition}
\begin{example}
    $R = \R, G = \{x^n \mid n \in \N\} \implies RG = \R[X]$
\end{example}
\begin{proposition}
$R[G]$ together with the ring homomorphism $\alpha \colon R \to R[G]$;
$r \mapsto r \cdot 1$ and the monoid homomorphism $\beta \colon 
G \to R[G]$; $g \mapsto 1 \cdot g$ 
enjoys the following universal property:
\[
  \alpha(r) \cdot \beta(g) = \beta(g) \cdot \alpha(r)
   \quad \forall r \in R, g \in G
\]
and if $(S,\alpha',\beta')$ is another such triple with
$\alpha'(r) \cdot \beta'(g) = \beta'(g) \cdot \alpha'(r)
   \quad \forall r \in R, g \in G$,
there is a unique monoid homomorphism $\gamma \colon R[G] \to S$
such that the following diagram commutes:
\[
    \begin{tikzcd}
        & S \\
        R \arrow[r,"\alpha"'] \arrow[ur,"\alpha'"] 
        & R[G] \arrow[u,"\gamma",dotted] 
        & G \arrow[l,"\beta"] \arrow[ul,"\beta'"']
    \end{tikzcd}
\]
\end{proposition}

Let $G \colon \cat{CRing} \to \cat{CMon}$ be the forgetful functor and
let $F \colon \cat{CMon} \to \cat{CRing}$ be the \textit{free monoid ring functor},
$M \mapsto \Z M$.
\begin{proposition}
    There is an adjoint situation \begin{tikzcd}
        \cat{CMon}
            \arrow[r, bend left = 25, "F"{name=D}]
            \arrow[r, leftarrow, bend right = 25, swap, "G"{name=C}]
              \arrow[d, from=D, to=C, phantom, "{\bot}"]
          & \cat{CRing}
    \end{tikzcd}
\end{proposition}
\begin{theorem}
    $\tau \colon \Z A \to \W(A)$ is a morphism of comonads.
\end{theorem}
