\chapter{Adjunctions}
Adjunctions are a way to compare categories. Two categories $\cat{A}$ and $\cat{B}$ are said to be \textit{isomorphic},
denoted $\cat{A} \cong \cat{B}$, if there exist functors $G \colon \cat{A} \to \cat{B},
F \colon \cat{B} \to \cat{A}$ with $FG = \id_A, GF = \id_B$. This condition is too strict to provide us
with many examples, which is why there is a different notion: $\cat{A}$ and $\cat{B}$ are said to be
\textit{equivalent},
denoted $\cat{A} \simeq \cat{B}$, if $FG \cong \id_A, GF \cong \id_B$ via natural isomorphisms
$\alpha \colon FG \to \id_A$, $\beta \colon GF \to \id_B$. Equivalent categories are essentially the same,
all categorical properties, like for example initial objects are preserved under equivalence. 
But there is an even less strict relation between categories, which was first introduced by Daniel Kan in 
\cite{kan} and which formalizes the notion of a free object:
consider the categories $\cat{Set}$ and $\cat{Vect_K}$ for a fixed field $K$.
The two categories can't be equivalent, since $\cat{Vect_K}$ has a zero object while
$\cat{Set}$ doesn't. But there still is a connection between them:

A vector space is a set with additional structure and linear maps are maps of sets which 
respect these structures. A different way to say this is that there is a forgetful functor 
$U \colon \cat{Vect_K} \to \cat{Set}.$ We can also go in the opposite direction, because
there is a "natural" way to make a set $X$ into a vector space:
\begin{itemize}
    \item We form the set $FX$ of all formal linear combinations of elements of $X$, i.e. 
    all elements of the form $\sum_{i=1}^{n}a_ix_i$
    \item we define addition and scalar multiplication by: 
    \[
        \lambda \cdot \bigl(\sum_{i=1}^{n}a_ix_i \bigr) := \sum_{i=1}^{n}(\lambda \cdot a_i) \cdot x_i
    \]
    \[
        \sum_{i=1}^{n}a_ix_i + \sum_{i=1}^{n}b_ix_i := \sum_{i=1}^{n}(a_i + b_i)x_i
    \]
(note that we can assume linear combinations in $FX$ to have the same length, since we can just add 0.)
\end{itemize}
This gives a vector space, the \textit{free vector space over X} which has $X$ as a basis. Now the universal property of a vector space states that
each map $X \to U(W)$ extends uniquely to a map $F(X) \to W$ and every map $F(X) \to W$ gives a map 
$X \to U(W)$ by restriction. This amounts to a bijection
\[
    \Hom[Vect_K](F(X),W) \cong \Hom[Set](X,U(W))
\]
which is also natural in a sense we will discuss later. The functors $F$ and $U$ form the first example and
one of the best known examples of an adjunction.

\section{Definition of Adjunctions}
We will start by giving two equivalent definitions of an adjunction, where the first one is especially useful 
when it comes to monads, while the second, more standard one, is easier to find examples.
\begin{proposition} \label{prop: equiv def of adj}
    Given two functors
    \begin{tikzcd}
        \cat{B} \ar[r,"F",shift left = .60ex]
          & 
        \cat{A} \ar[l,"G",shift left = .60ex]
    \end{tikzcd}
    the following are equivalent: 
    \begin{enumerate}[(a)]
    \item There are natural transformations $\eta\colon \id_B \Rightarrow GF$ and $\eps\colon FG \Rightarrow \id_A$ 
    such that for all objects $a$ of $\cat{A}$, $b$ of $\cat{B}$
    the following two diagrams commute:
    \begin{equation}\label{eq: triangle identity of adjunction}
        \begin{tikzcd}\tag{triangle identity}
            F(b) \arrow[rd,"\id_{F(b)}"'] \arrow[r, "F(\eta_b)"] & FGF(b) \arrow[d,"\eps_{F(b)}"] \\
                                        & F(b)
        \end{tikzcd}
        \qquad
        \begin{tikzcd} 
    G(a) \arrow[rd,"id_{G(a)}"'] \arrow[r, "\eta_{G(a)}"] & GFG(a) \arrow[d,"G(\eps_a)"] \\
                                        & G(a)
        \end{tikzcd}
    \end{equation}
    \item There is a bijection 
    \[
    \phi_{a,b}\colon\Hom[A](F(b),a)\cong \Hom[B](b,G(a))
    \]
    for all objects $a$ of $\cat{A}$ and $b$ of $\cat{B}$, which is natural in $a$ and $b$.
    \end{enumerate}
\end{proposition}
Naturality here means that 
for $p\colon a\to a'$ and for $q\colon b \to b'$ the following two diagrams commute:
\begin{figure}[H]
\mbox{}\hfill 
\begin{subfigure}{0.47\textwidth}
\centering
%\phantomsection\label{name1}
\begin{tikzcd}
    \Hom[A](F(b),a) \arrow[r,"\phi_{a,b}"] \arrow[d,"p \circ \_"] 
      & \Hom[B](b,G(a)) \arrow[d,"G(p)\circ \_"] \\
    \Hom[A](F(b),a') \arrow[r,"\phi_{a',b}"]
      & \Hom[B](b,G(a'))
\end{tikzcd}
%\caption*{(cap1)}
\end{subfigure}
\hfill
\begin{subfigure}{0.47\textwidth}
\centering
%\phantomsection\label{name2}
\begin{tikzcd}
    \Hom[A](F(b'),a) \arrow[r,"\phi_{a,b'}"] \arrow[d,"\_ \circ F(q)"] 
      & \Hom[B](b',G(a)) \arrow[d,"\_ \circ q"] \\
    \Hom[A](F(b),a) \arrow[r,"\phi_{a,b}"]
      & \Hom[B](b,G(a))
\end{tikzcd}
%\caption*{(cap2)}
\end{subfigure}
\hfill \mbox{}
\end{figure}

\begin{bigproof}
    $(a)\Rightarrow (b):$
    Define 
    \[
        \phi_{a,b}\colon \Hom[A](F(b),a)\to \Hom[B](b,G(a)) 
        \quad \text{ by } \quad \phi_{a,b}(g) = G(g) \circ \eta_b 
        \colon b \to G(a)
    \]
    \[
        \psi_{a,b} \colon \Hom[B](b,G(a)) \to \Hom[A](F(b),a)
        \quad \text{ by } \quad \psi_{a,b}(f) = \eps_a \circ F(f) 
        \colon F(b) \to a
    \]
    for $g\colon F(b)\to a$, $f\colon b \to G(a)$.
\begin{claim*}
$\phi \circ \psi = \id$
\end{claim*}
\begin{smallproof}
Let $f \colon b \to G(a)$.
\begin{align*}
    \phi(\psi(f)) &= \phi(\eps_a \circ F(f)) \tag{Definition of $\psi$} \\
    &= G(\eps_a \circ F(f)) \circ \eta_b \tag{Definition of $\phi$}\\
    &= G(\eps_a) \circ G(F(f)) \circ \eta_b \tag{Functoriality of $G$}\\
    &= G(\eps_a) \circ \eta_{G(a)} \circ f \tag{Naturality of $\eta$} \\
    &= \id_{G(a)} \circ f = f \tag{right~\ref{eq: triangle identity of adjunction}}
\end{align*}
\end{smallproof}
\begin{claim*}
    $\psi \circ \phi = \id$
\end{claim*}
\begin{smallproof}
\begin{align*}
        \psi(\phi(g)) &= \psi (G(g) \circ \eta_b) \tag{Definition of $\phi$}\\
        &= \eps_a \circ F(G(g) \circ \eta_b) \tag{Definition of $\psi$}\\
        &= \eps_a \circ F(G(g)) \circ F(\eta_b) \tag{Functoriality of $F$}\\
        &= g \circ \eps_{F(b)} \circ F(\eta_b) \tag{Naturality of $\eps$}\\
        &= g \circ \id_{F(b)} = g \tag{left~\ref{eq: triangle identity of adjunction}}
\end{align*}
\end{smallproof}
\begin{claim*}
    $\phi_{a,b}$ is natural in $a$.
\end{claim*}
\begin{smallproof}
    Let $p \colon a \to a'$. Then by functoriality of $G$ we have:
    \[
      G(p) \circ \phi_{a,b}(g)=  G(p) \circ G(g) \circ \eta_b =
      G(p \circ g) \circ \eta_b = \phi_{a',b}(p \circ g).
    \]
\end{smallproof}
\begin{claim*}
    $\phi_{a,b}$ is natural in $b$.
\end{claim*}
\begin{smallproof}
    Let $q \colon b \to b'$. Then by functoriality of $G$ and naturality of $\eta$ we have: 
    \[
       \phi_{a,b}(g \circ F(q)) = G(g \circ F(q)) \circ \eta_b = G(g) \circ GF(q) \circ \eta_b
       = G(g) \circ \eta_{b'} \circ q = \phi_{a,b'}(g) \circ q
     \]
\end{smallproof}
$(a)\Leftarrow (b):$ Define 
\[
    \eta \colon id_{\cat{B}} \Rightarrow GF \quad \text{ by } \quad \eta_b := \phi_{F(b),b}(\id_{F(b)})
    \colon b \to GF(b) \\
\]
\[
    \eps \colon FG \Rightarrow id_{\cat{A}} \quad \text{ by } \quad \eps_a := \psi_{a,G(a)}(\id_{G(a)})
    \colon FG(a) \to a
\]
\begin{claim*}
    $\eta$ is a natural transformation.
\end{claim*}
\begin{smallproof}
    For $q \colon b \to b'$ we need to show that 
    \[
        \begin{tikzcd}
            b \arrow[r,"q"] \arrow[d,"\eta_b"] 
              & b' \arrow[d,"\eta_{b'}"] \\
            GF(b) \arrow[r,"GF(q)"]
              & GF(b')
        \end{tikzcd}
    \]
    commutes. But using the naturality of $\phi$ (applied to $q \colon b \to b'$ and
    $F(q) \colon Fb \to Fb'$), we get
    \begin{align*}
        \eta_{b'} \circ q &= \phi(\id_{Fb'}) \circ q = \phi(\id_{Fb'} \circ F(q))  = \phi(F(q) \circ \id_{Fb}) \\
        &= GF(q) \circ \phi(\id_{F(b)}) = GF(q) \circ \eta_b.
    \end{align*}
\end{smallproof}
\begin{claim*}
    $\eps$ is a natural transformation.
\end{claim*}
\begin{smallproof}
    For $p \colon a \to a'$ we need to show that
    \[
      \begin{tikzcd}
          FGa \arrow[r,"FG(p)"] \arrow[d,"\eps_a"] 
            & FGa' \arrow[d,"\eps_{a'}"] \\
          a \arrow[r,"p"]
            & a'
      \end{tikzcd}  
    \]
    commutes. Since $\phi$ is natural, $\psi$ is natural as well, meaning that the diagrams above
    where $\phi$ is replaced by $\psi$ commute. Using the naturality of $\psi$ in $a$ (applied to $p \colon a \to a'$)
    and $b$ (applied to $G(p) \colon Ga \to Ga'$), we get
    \begin{align*}
        p \circ \eps_a &= p \circ \psi(\id_{Ga}) = \psi(G(p) \circ \id_{Ga}) = \psi(\id_{Ga'} \circ G(p)) \\
        &= \psi(\id_{Ga'}) \circ FG(p) = \eps_{a'} \circ FG(p).
    \end{align*}
\end{smallproof}
\begin{claim*} 
    $\eta$ and $\eps$ satisfy the triangle identities.
\end{claim*}
\begin{smallproof}
    \[
        \id_{F(b)} = \psi(\phi(\id_{F(b)})) = \psi(\eta_b) 
        = \psi(\eta_b \circ \id_b) = \psi(\id_b) \circ F(\eta_b) = \eps_{F(b)} \circ F(\eta_b)
    \]
    \[
       \id_{G(a)} = \phi(\psi(\id_{G(a)})) = \phi(\eps_a) = \phi(\id_a \circ \eps_a) 
       = G(\eps_a) \circ \phi(\id_a) = G(\eps_a) \circ \eta_a
    \]
    using the definitions of $\eta$, $\eps$ and the naturality of $\phi$ and $\psi$.
\end{smallproof}
\end{bigproof}
\begin{definition}[Adjunction]
    Let $\cat{A}$ and $\cat{B}$ be categories. 
    Then functors
    $F \colon \cat{B} \to \cat{A}$, $G \colon \cat{A} \to \cat{B}$
    are said to form an \textit{adjunction between $\cat{A}$ and $\cat{B}$},
    if $F$ and $G$ satisfy the equivalent conditions of~\ref{prop: equiv def of adj}. 
    In this case $F$ is called \textit{left-adjoint} to $G$ and $G$ is called \textit{right-adjoint}
    to $F$. \\ \\
    We will denote the adjunction either by $F \! \adj_{\phi}\!\! G \colon \adjar{B}{A}$ or by 
    $F {\ }_{\eps}^{\text{\tiny{$\eta$}}}\!\!\!\!\adj G \colon \adjar{B}{A}$,
    sometimes even just $F \adj G$ or 
    \begin{tikzcd}
        \cat{B}
            \arrow[r, bend left = 25, "F"{name=D}]
            \ar[r,"\bot",phantom,description]
          & \cat{A} \arrow[l ,bend left = 25,"G"{name=C}]
    \end{tikzcd}, depending on the context. The natural transformations
    $\eta$ and $\eps$ are called \textit{unit} respectively \textit{counit}
    of the adjunction.

\end{definition}
The following is another reason why adjunctions are a very useful concept:
\begin{remark}[{proven for example in \cite[{Chapter 4.5}]{riehl}}]
    Let $F \adj G$ be an adjunction.
    Then
    \begin{enumerate}[(a)]
        \item G preserves limits
        \item F preserves colimits.
    \end{enumerate}
\end{remark}

\section{Examples}

\begin{example}[free-forgetful adjunction]
    The example from the introduction is an example of a so called \textit{free-forgetful adjunction},
    where a functor is left-adjoint to a forgetful functor.
    These adjunctions arise for almost any algebraic structure: groups, semigroups, rings, monoids, etc.
    For example the \textit{free group} $F_S$ of a set $S$ can be constructed as the set
    of all reduced words in the alphabet $S \cup S^{-1}$, with the group
    multiplication being concatenation followed by reduction. This construction is
    again functorial and enjoys a universal property, which can be reformulated as
    \[
        \Hom[Grp](F_S,G) \cong \Hom[Set](S,U(G))
    \]
    for every set $S$ and every group $G$, where $U$ is the forgetful functor $\cat{Grp} \to \cat{Set}$.
    Thus we again have an adjunction $F \adj U \colon \adjar{Set}{Grp}$.
\end{example}
\begin{example}[Tensor-Hom-Adjunction]
    If $A$ is a ring, then by the universal property of the tensor product
    of $A$-modules, for a fixed $A$-module $N$, there is a natural bijection
    \[
        \Hom[A](M \tensor_A N,P) \cong \Hom[A](M,\Hom[A](N,P))
    \]
    for all $A$-modules $M$ and $P$.
    This implies that the tensor-product is right-exact, since as a left-adjoint,
    it preserves colimits, so in particular cokernels. 
\end{example}
\begin{example}[Galois connection]
    Let $(\cat{A},\leq)$ and $(\cat{B},\leq)$ be two partially ordered sets.
    A \textit{monotone Galois connection} consists of two monotone functions
    $F \colon \cat{B} \to \cat{A}$ and $G \colon \cat{A} \to \cat{B}$ such that
    $\forall a \in \cat{A}, b \in \cat{B}$, we have
    \begin{equation}
        Fb \leq a \iff b \leq Ga \label{eq: monotone galois} \tag{$\ast$}
    \end{equation}
    An \textit{antitone Galois connection} consists of two antitone (order-reversing)
    functions $F \colon \cat{B} \to \cat{A}$ and $G \colon \cat{A} \to \cat{B}$
    such that $\forall a \in \cat{A}, b \in \cat{B}$, we have
    \begin{equation}
        a \leq Fb \iff b \leq Ga \label{eq: antitone galois} \tag{$\ast \ast$}
    \end{equation}
    Now every partially ordered set $(\cat{P},\leq)$ can be viewed as a category
    with objects the elements of $\cat{P}$ and a morphism between $x$ and $y$ if and only
    if $x \leq y$. Then if $F$ and $G$ are monotone, they are functors and (\ref{eq: monotone galois})
    can be restated as
    \[
        \Hom[A](Fb,a) \cong \Hom[B](b,Ga)
    \]
    thus a monotone Galois connection froms and adjunction between the categories
    $\cat{A}$ and $\cat{B}$ (the naturality diagrams alle commute, because there is at 
    most one morphism between objects). 
    For the antitone case, (\ref{eq: antitone galois}) gives us
    \begin{align*}
        &\Hom[A](a,Fb) \cong \Hom[B](b,Ga) \\
        \iff &\Hom[A^{op}](Fb,a) \cong \Hom[B](b,Ga)
    \end{align*}
    and since $F$ and $G$ are antitone, we can view them as (covariant) functors
    $\adjar{B}{A^{op}}$. Thus an antitone Galois connection forms an adjunction between $\cat{A^{op}}$
    and $\cat{B}$. 
    Galois connections appear all over mathematics, here are a few examples: \\
    \begin{enumerate}
        \item (Convex sets):
        Let $\cat{C}:=\set{U \subseteq \R^n \mid U \text{ is convex}}$ and
        $\cat{S} := \set{V \subseteq \R^n}$, both partially ordered by inclusion.
        Define two monotonic functions $F \colon \cat{S} \to \cat{C}$ and
        $G \colon \cat{C} \to \cat{S}$ by $F(V) = \hull{V}$, where $\hull{\cdot}$
        denotes the convex hull operator, and $G(U) = U$. Then we have
        \[
          \hull{V} \subseteq U \iff V \subseteq U  
        \]
        hence a monotone Galois connection.
        
        \item (Algebraic geometry): 
        Let $k$ be an algebraically closed field, $n \in \N$. Then for
        $\mathcal{R} := \set{I \subseteq k[X_1,\dots,X_n]}$ and
        $\mathcal{A} := \set{M \subseteq \A^n(k)}$ there are inclusion-reversing
        maps \begin{tikzcd}
            \mathcal{R}
                \arrow[r,"\mathcal{V}"{name=D},shift left = .60ex]
              & \mathcal{A} \arrow[l,"\mathcal{I}"{name=C},shift left = .60ex]
        \end{tikzcd}
        where $\mathcal{V}(I)$ is the \textit{zero set} of $I$ and $\mathcal{I}(M)$
        is the \textit{vanishing ideal} of $M$ (\textit{Hilbert's Nullstellensatz} deals 
        precisely with this correspondence). In particular we have
        \[
          M \subseteq \mathcal{V}(I) \iff I \subseteq \mathcal{I}(M)  
        \]
        hence an antitone Galois connection.

        \item  (Galois theory):
        Let $L/K$ be a field extension, let $\cat{A} := \set{K \subseteq M \subseteq L \mid M \text{ is a field}}$
        and $\cat{B} := \set{G \subseteq Gal(L/K)}$ where 
        $Gal(L/K)$ is the group of automorphisms on $L$ that fix $K$. 
        For a subgroup $G \subseteq Gal(L/K)$ let $L^G$ be the \textit{fixed field}
        of $G$. Then there is an antitone Galois connection given by
        \[
            E \mapsto Gal(L/E), \ G \mapsto L^G.
        \]

    \end{enumerate}
\begin{example}[Coproduct $\adj$ $\Delta$ $\adj$ Product]
    Even universal construction such as (co-)products can be seen as adjunctions:
    For a category $\cat{C}$ that has finite products and coproducts, we have the 
    \textit{diagonal functor} $\Delta \colon \cat{C} \to \cat{C} \times \cat{C}$,
    $\Delta X = (X,X)$, $\Delta(f) = (f,f)$ as well as the \textit{product functor}
    $Prod \colon \cat{C} \times \cat{C} \to \cat{C}$, $Prod(X,Y) = X \times Y$,
    $Prod((f,g) \colon (X_1,Y_1) \to (X_2,Y_2)) = f \times g \colon X_1 \times Y_1
    \to X_2 \times Y_2$ and the \textit{coproduct functor} 
    $Coprod \colon \cat{C} \times \cat{C} \to \cat{C}$, $Prod(X,Y) = X \sqcup Y$,
    $Coprod((f,g) \colon (X_1,Y_1) \to (X_2,Y_2)) = f \sqcup g \colon X_1 \sqcup Y_1
    \to X_2 \sqcup Y_2$.
    Then the universal property of the product states that 
    \[
        \Hom[{C \times C}]((A,A),(X,Y)) \cong \Hom[C](A,X \times Y)
    \]
    for every $A \in \cat{C}$, which shows $\Delta \adj Prod$. 
    Analogous one sees that $Coprod \adj \Delta$.
\end{example}
\end{example}