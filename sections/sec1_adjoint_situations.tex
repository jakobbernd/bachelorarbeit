
\chapter{Adjoint situations}

\todo{einleitender Satz}
Moreover, adjunctions provide us with many (technically even all) 
examples of monads and comonads, as we will later see.
\begin{proposition} \label{prop: equiv def of adj}
    Given two functors
    \begin{tikzcd}
        \cat{B} \ar[r,"F",shift left = .60ex]
          & 
        \cat{A} \ar[l,"G",shift left = .60ex]
    \end{tikzcd}
    the following are equivalent: 
    \begin{enumerate}[(a)]
    \item There are natural transformations $\eta\colon \id_B \Rightarrow GF$ and $\eps\colon FG \Rightarrow \id_A$ 
    such that for all objects $a$ of $\cat{A}$, $b$ of $\cat{B}$
    the following two diagrams commute:
    \begin{equation}\label{eq:triangle identity of adjunction}
        \begin{tikzcd}\tag{triangle identity}
            F(b) \arrow[rd,"\id_{F(b)}"'] \arrow[r, "F(\eta_b)"] & FGF(b) \arrow[d,"\eps_{F(b)}"] \\
                                        & F(b)
        \end{tikzcd}
        \qqq
        \begin{tikzcd} 
    G(a) \arrow[rd,"id_{G(a)}"'] \arrow[r, "\eta_{G(a)}"] & GFG(a) \arrow[d,"G(\eps_a)"] \\
                                        & G(a)
        \end{tikzcd}
    \end{equation}
    \item There is a bijection 
    \[
    \phi_{a,b}\colon\Hom[A](F(b),a)\cong \Hom[B](b,G(a))
    \]
    for all objects $a$ of $\cat{A}$ and $b$ of $\cat{B}$, which is natural in $a$ and $b$.
    \end{enumerate}
\end{proposition}
Naturality here means that 
for $p\colon a\to a'$ and for $q\colon b \to b'$ the following two diagrams commute:
\begin{figure}[H]
\centering
\begin{subfigure}{0.4\textwidth}
\centering
%\phantomsection\label{name1}
\begin{tikzcd}
    \Hom[A](F(b),a) \arrow[r,"\phi_{a,b}"] \arrow[d,"p \circ \_"] 
      & \Hom[B](b,G(a)) \arrow[d,"G(p)\circ \_"] \\
    \Hom[A](F(b),a') \arrow[r,"\phi_{a',b}"]
      & \Hom[B](b,G(a'))
\end{tikzcd}
%\caption*{(cap1)}
\end{subfigure}
\hspace{2em}
\begin{subfigure}{0.4\textwidth}
\centering
%\phantomsection\label{name2}
\begin{tikzcd}
    \Hom[A](F(b'),a) \arrow[r,"\phi_{a,b'}"] \arrow[d,"\_ \circ F(q)"] 
      & \Hom[B](b',G(a)) \arrow[d,"\_ \circ q"] \\
    \Hom[A](F(b),a) \arrow[r,"\phi_{a,b}"]
      & \Hom[B](b,G(a))
\end{tikzcd}
%\caption*{(cap2)}
\end{subfigure}
\end{figure}


\begin{bigproof}
    $(a)\implies (b):$ \\
    define 
    \[
        \phi_{a,b}\colon \Hom[A](F(b),a)\to \Hom[B](b,G(a)) 
        \quad \text{ by } \quad \phi_{a,b}(g) = G(g) \circ \eta_b 
        \colon b \to G(a)
    \]
    \[
        \psi_{a,b} \colon \Hom[B](b,G(a)) \to \Hom[A](F(b),a)
        \quad \text{ by } \quad \psi_{a,b}(f) = \eps_a \circ F(f) 
        \colon F(b) \to a
    \]
    for $g\colon F(b)\to a$, $f\colon b \to G(a)$.
\begin{claim}
$\phi \circ \psi = \id$
\end{claim}
\begin{smallproof}
Let $f \colon b \to G(a)$.
\begin{align*}
    \phi(\psi(f)) &= \phi(\eps_a \circ F(f)) \tag{Definition of $\psi$} \\
    &= G(\eps_a \circ F(f)) \circ \eta_b \tag{Definition of $\phi$}\\
    &= G(\eps_a) \circ G(F(f)) \circ \eta_b \tag{Functoriality of $G$}\\
    &= G(\eps_a) \circ \eta_{G(a)} \circ f \tag{Naturality of $\eta$} \\
    &= \id_{G(a)} \circ f = f \tag{right~\ref{eq:triangle identity of adjunction}}
\end{align*}
\end{smallproof}
\begin{claim}
    $\psi \circ \phi = \id$
\end{claim}
\begin{smallproof}
\begin{align*}
        \psi(\phi(g)) &= \psi (G(g) \circ \eta_b) \tag{Definition of $\phi$}\\
        &= \eps_a \circ F(G(g) \circ \eta_b) \tag{Definition of $\psi$}\\
        &= \eps_a \circ F(G(g)) \circ F(\eta_b) \tag{Functoriality of $F$}\\
        &= g \circ \eps_{F(b)} \circ F(\eta_b) \tag{Naturality of $\eps$}\\
        &= g \circ \id_{F(b)} = g \tag{left~\ref{eq:triangle identity of adjunction}}
\end{align*}
\end{smallproof}
\begin{claim}
    $\phi_{a,b}$ is natural in $a$.
\end{claim}
\begin{smallproof}
    Let $p \colon a \to a'$. Then by functoriality of $G$ we have:
    \[
       G(p) \circ G(g) \circ \eta_b = G(p \circ g) \circ \eta_b.
    \]
\end{smallproof}
\begin{claim}
    $\phi_{a,b}$ is natural in $b$.
\end{claim}
\begin{smallproof}
    Let $q \colon b \to b'$. Then by functoriality of $G$ and naturality of $\eta$ we have: 
    \[
       G(g \circ F(q)) \circ \eta_b = G(g) \circ GF(q) \circ \eta_b
       = G(g) \circ \eta_{b'} \circ q
     \]
\end{smallproof}
$(a)\Longleftarrow (b):$ Define 
\[
    \eta \colon id_{\cat{B}} \Rightarrow GF \quad \text{ by } \quad \eta_b := \phi_{F(b),b}(\id_{F(b)})
    \colon b \to GF(b) \\
\]
\[
    \eps \colon FG \Rightarrow id_{\cat{A}} \quad \text{ by } \quad \eps_a := \psi_{a,G(a)}(\id_{G(a)})
    \colon FG(a) \to a
\]
\begin{claim}
    $\eta$ is a natural transformation.
\end{claim}
\begin{smallproof}
    For $p \colon b \to b'$ we need to show that 
    \[
        \begin{tikzcd}
            b \arrow[r,"q"] \arrow[d,"\eta_b"] 
              & b' \arrow[d,"\eta_{b'}"] \\
            GF(b) \arrow[r,"GF(q)"]
              & GF(b')
        \end{tikzcd}
    \]
    commutes, which means $\phi_{F(b),b}(\id_{F(b)}) \circ p 
    = GF(p) \circ \phi_{F(b'),b'}(\id_{F(b')})$. \todo{?}
    But 
    \[
        \phi(\id) \circ p = \phi(F(p)) = GF(p) \circ \phi
    \]
\end{smallproof}
\begin{claim}
    $\eps$ is a natural transformation.
\end{claim}
\begin{smallproof}
    \todo{?}
\end{smallproof}
\begin{claim} 
    $\eta$ and $\eps$ satisfy the triangle identities.
\end{claim}
\begin{smallproof}
    \[
        \id_{F(b)} = \psi(\phi(\id_{F(b)})) = \psi(\eta_b) 
        = \psi(\eta_b \circ \id_b) = \eps_{F(b)} \circ F(\eta_b)
    \]
    \[
       \id_{G(a)} = \phi(\psi(\id_{G(a)})) = \phi(\eps_a) = \phi(id_a) 
    \]
\end{smallproof}
\end{bigproof}
\begin{definition}[Adjunction]
    Let $\cat{A}$ and $\cat{B}$ be categories. 
    We say that functors
    $F \colon \cat{B} \to \cat{A}$, $G \colon \cat{A} \to \cat{B}$
    form an \textit{adjunction between $\cat{A}$ and $\cat{B}$},
    if $F$ and $G$ satisfy the equivalent conditions of~\ref{prop: equiv def of adj}. 
    We then say that $F$ is \textit{left-adjoint} to $G$ and $G$ is \textit{right-adjoint}
    to $F$.
\end{definition}
\begin{remark}
    We will denote the adjunction either by $F \! \adj_{\phi}\!\! G \colon \adjar{B}{A}$ or by 
    $F {\ }_{\eps}^{\text{\tiny{$\eta$}}}\!\!\!\!\adj G \colon \adjar{B}{A}$,
    sometimes even just $F \adj G$, depending on the context.
\end{remark}

\begin{remark}
    Let $F \adj G$ be an adjunction.
    Then
    \begin{enumerate}
        \item G preserves limits
        \item F preserves colimits.
    \end{enumerate}
\end{remark}

\begin{example}[Galois connection]\todo{those are DUAL adjunctions!}
    blablabla examples include:
    \begin{enumerate}
        \item (Fundamental theorem of Galois theory): 
        this example.
        \item  (Algebraic geometry):
        that example.
    \end{enumerate}

\end{example}
\begin{example}[Coproduct $\adj$ $\Delta$ $\adj$ Product]
    this.
\end{example}
\begin{example}[free-forgetful adjunction]
    that.
\end{example}
\begin{example}[Tensor-Hom-Adjunction]
    There is a natural bijection
    \[
        \Hom[A](M \tensor_A N,P) \cong \Hom[A](M,\Hom[A](N,P))
    \]
    This implies that the tensor-product is right-exact, since it preserves cokernels.
\end{example}
