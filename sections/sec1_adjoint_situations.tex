\chapter{Adjunctions}
Let $\cat{A}, \cat{B}$ be two categories. We say that $\cat{A}$ and $\cat{B}$ are \textit{isomorphic},
denoted $\cat{A} \cong \cat{B}$, if there exist functors $G \colon \cat{A} \to \cat{B},
F \colon \cat{B} \to \cat{A}$ with $FG = \id_A, GF = \id_B$. This condition is too strict to provide us
with many examples, which is why there is a different notion: We say that $\cat{A}$ and $\cat{B}$ 
are \textit{equivalent},
denoted $\cat{A} \simeq \cat{B}$, if $FG \cong \id_A, GF \cong \id_B$ via natural isomorphisms
$\alpha \colon FG \to \id_A$, $\beta \colon GF \to \id_B$. Equivalent categories are essentially the same,
all categorical properties, like for example initial objects are preserved under equivalence. 
But there is an even less strict relation between categories, which was first introduced by Daniel Kan in 
\cite{kan} and which is a powerful concept, because it arises so often:
Consider the categories $\cat{Set}$ and $\cat{Vect_K}$ for a fixed field $K$.
The two categories can't be equivalent, since $\cat{Vect_K}$ has a zero object while
$\cat{Set}$ doesn't. But there still is a connection between them:

A vector space is a set with additional structure and linear maps are maps of sets which 
respect these structures. A different way to say this is that there is a forgetful functor 
$U \colon \cat{Vect_K} \to \cat{Set}.$ We can also go in the opposite direction, because
there is a "natural" way to make a set $X$ into a vector space:
\begin{itemize}
    \item We form the set $FX$ of all formal linear combinations of elements of $X$, i.e. 
    all elements of the form $\sum_{i=1}^{n}a_ix_i$
    \item we define addition and scalar multiplication by: 
    \[
        \lambda \cdot \bigl(\sum_{i=1}^{n}a_ix_i \bigr) := \sum_{i=1}^{n}(\lambda \cdot a_i) \cdot x_i
    \]
    \[
        \sum_{i=1}^{n}a_ix_i + \sum_{i=1}^{n}b_ix_i := \sum_{i=1}^{n}(a_i + b_i)x_i
    \]
(note that we can assume linear combinations in $FX$ to have the same length, since we can just add 0.)
\end{itemize}
This gives a vector space which has $X$ as a basis. Now the universal property of a vector space states that
each map $X \to U(W)$ extends uniquely to a map $F(X) \to W$ and every map $F(X) \to W$ gives a map 
$X \to U(W)$ by restriction. This amounts to a bijection
\[
    \Hom[Vect_K](F(X),W) \cong \Hom[Set](X,U(W))
\]
which is also natural in a sense we will discuss later. The functors $F$ and $U$ form what is called
a \textit{free-forgetful adjunction}, which is the first example of an adjunction.

\section{Definition of adjunctions}
We will start by giving two equivalent definitions of an adjunction, where the first one is especially useful 
when it comes to monads, while the second, more standard one, is easier to find examples.
\newpage
\begin{proposition} \label{prop: equiv def of adj}
    Given two functors
    \begin{tikzcd}
        \cat{B} \ar[r,"F",shift left = .60ex]
          & 
        \cat{A} \ar[l,"G",shift left = .60ex]
    \end{tikzcd}
    the following are equivalent: 
    \begin{enumerate}[(a)]
    \item There are natural transformations $\eta\colon \id_B \Rightarrow GF$ and $\eps\colon FG \Rightarrow \id_A$ 
    such that for all objects $a$ of $\cat{A}$, $b$ of $\cat{B}$
    the following two diagrams commute:
    \begin{equation}\label{eq: triangle identity of adjunction}
        \begin{tikzcd}\tag{triangle identity}
            F(b) \arrow[rd,"\id_{F(b)}"'] \arrow[r, "F(\eta_b)"] & FGF(b) \arrow[d,"\eps_{F(b)}"] \\
                                        & F(b)
        \end{tikzcd}
        \qqq
        \begin{tikzcd} 
    G(a) \arrow[rd,"id_{G(a)}"'] \arrow[r, "\eta_{G(a)}"] & GFG(a) \arrow[d,"G(\eps_a)"] \\
                                        & G(a)
        \end{tikzcd}
    \end{equation}
    \item There is a bijection 
    \[
    \phi_{a,b}\colon\Hom[A](F(b),a)\cong \Hom[B](b,G(a))
    \]
    for all objects $a$ of $\cat{A}$ and $b$ of $\cat{B}$, which is natural in $a$ and $b$.
    \end{enumerate}
\end{proposition}
Naturality here means that 
for $p\colon a\to a'$ and for $q\colon b \to b'$ the following two diagrams commute:
\begin{figure}[H]
\centering
\begin{subfigure}{0.4\textwidth}
\centering
%\phantomsection\label{name1}
\begin{tikzcd}
    \Hom[A](F(b),a) \arrow[r,"\phi_{a,b}"] \arrow[d,"p \circ \_"] 
      & \Hom[B](b,G(a)) \arrow[d,"G(p)\circ \_"] \\
    \Hom[A](F(b),a') \arrow[r,"\phi_{a',b}"]
      & \Hom[B](b,G(a'))
\end{tikzcd}
%\caption*{(cap1)}
\end{subfigure}
\hspace{2em}
\begin{subfigure}{0.4\textwidth}
\centering
%\phantomsection\label{name2}
\begin{tikzcd}
    \Hom[A](F(b'),a) \arrow[r,"\phi_{a,b'}"] \arrow[d,"\_ \circ F(q)"] 
      & \Hom[B](b',G(a)) \arrow[d,"\_ \circ q"] \\
    \Hom[A](F(b),a) \arrow[r,"\phi_{a,b}"]
      & \Hom[B](b,G(a))
\end{tikzcd}
%\caption*{(cap2)}
\end{subfigure}
\end{figure}


\begin{bigproof}
    $(a)\implies (b):$ \\
    define 
    \[
        \phi_{a,b}\colon \Hom[A](F(b),a)\to \Hom[B](b,G(a)) 
        \quad \text{ by } \quad \phi_{a,b}(g) = G(g) \circ \eta_b 
        \colon b \to G(a)
    \]
    \[
        \psi_{a,b} \colon \Hom[B](b,G(a)) \to \Hom[A](F(b),a)
        \quad \text{ by } \quad \psi_{a,b}(f) = \eps_a \circ F(f) 
        \colon F(b) \to a
    \]
    for $g\colon F(b)\to a$, $f\colon b \to G(a)$.
\begin{claim}
$\phi \circ \psi = \id$
\end{claim}
\begin{smallproof}
Let $f \colon b \to G(a)$.
\begin{align*}
    \phi(\psi(f)) &= \phi(\eps_a \circ F(f)) \tag{Definition of $\psi$} \\
    &= G(\eps_a \circ F(f)) \circ \eta_b \tag{Definition of $\phi$}\\
    &= G(\eps_a) \circ G(F(f)) \circ \eta_b \tag{Functoriality of $G$}\\
    &= G(\eps_a) \circ \eta_{G(a)} \circ f \tag{Naturality of $\eta$} \\
    &= \id_{G(a)} \circ f = f \tag{right~\ref{eq: triangle identity of adjunction}}
\end{align*}
\end{smallproof}
\begin{claim}
    $\psi \circ \phi = \id$
\end{claim}
\begin{smallproof}
\begin{align*}
        \psi(\phi(g)) &= \psi (G(g) \circ \eta_b) \tag{Definition of $\phi$}\\
        &= \eps_a \circ F(G(g) \circ \eta_b) \tag{Definition of $\psi$}\\
        &= \eps_a \circ F(G(g)) \circ F(\eta_b) \tag{Functoriality of $F$}\\
        &= g \circ \eps_{F(b)} \circ F(\eta_b) \tag{Naturality of $\eps$}\\
        &= g \circ \id_{F(b)} = g \tag{left~\ref{eq: triangle identity of adjunction}}
\end{align*}
\end{smallproof}
\begin{claim}
    $\phi_{a,b}$ is natural in $a$.
\end{claim}
\begin{smallproof}
    Let $p \colon a \to a'$. Then by functoriality of $G$ we have:
    \[
       G(p) \circ G(g) \circ \eta_b = G(p \circ g) \circ \eta_b.
    \]
\end{smallproof}
\begin{claim}
    $\phi_{a,b}$ is natural in $b$.
\end{claim}
\begin{smallproof}
    Let $q \colon b \to b'$. Then by functoriality of $G$ and naturality of $\eta$ we have: 
    \[
       G(g \circ F(q)) \circ \eta_b = G(g) \circ GF(q) \circ \eta_b
       = G(g) \circ \eta_{b'} \circ q
     \]
\end{smallproof}
$(a)\Longleftarrow (b):$ Define 
\[
    \eta \colon id_{\cat{B}} \Rightarrow GF \quad \text{ by } \quad \eta_b := \phi_{F(b),b}(\id_{F(b)})
    \colon b \to GF(b) \\
\]
\[
    \eps \colon FG \Rightarrow id_{\cat{A}} \quad \text{ by } \quad \eps_a := \psi_{a,G(a)}(\id_{G(a)})
    \colon FG(a) \to a
\]
\begin{claim}
    $\eta$ is a natural transformation.
\end{claim}
\begin{smallproof}
    For $p \colon b \to b'$ we need to show that 
    \[
        \begin{tikzcd}
            b \arrow[r,"q"] \arrow[d,"\eta_b"] 
              & b' \arrow[d,"\eta_{b'}"] \\
            GF(b) \arrow[r,"GF(q)"]
              & GF(b')
        \end{tikzcd}
    \]
    commutes, which means $\phi_{F(b),b}(\id_{F(b)}) \circ p 
    = GF(p) \circ \phi_{F(b'),b'}(\id_{F(b')})$. \todo{finish}
    But 
    \[
        \phi(\id) \circ p = \phi(F(p)) = GF(p) \circ \phi
    \]
\end{smallproof}
\begin{claim}
    $\eps$ is a natural transformation.
\end{claim}
\begin{smallproof}
    \todo{finish}
\end{smallproof}
\begin{claim} 
    $\eta$ and $\eps$ satisfy the triangle identities.
\end{claim}
\begin{smallproof}
    \[
        \id_{F(b)} = \psi(\phi(\id_{F(b)})) = \psi(\eta_b) 
        = \psi(\eta_b \circ \id_b) = \eps_{F(b)} \circ F(\eta_b)
    \]
    \[
       \id_{G(a)} = \phi(\psi(\id_{G(a)})) = \phi(\eps_a) = \phi(id_a) 
    \]
\end{smallproof}
\end{bigproof}
\todo{or yoneda lemma, see p.193 Brandenburg}
\begin{definition}[Adjunction]
    Let $\cat{A}$ and $\cat{B}$ be categories. 
    We say that functors
    $F \colon \cat{B} \to \cat{A}$, $G \colon \cat{A} \to \cat{B}$
    form an \textit{adjunction between $\cat{A}$ and $\cat{B}$},
    if $F$ and $G$ satisfy the equivalent conditions of~\ref{prop: equiv def of adj}. 
    We then say that $F$ is \textit{left-adjoint} to $G$ and $G$ is \textit{right-adjoint}
    to $F$.
\end{definition}
\begin{remark}
    We will denote the adjunction either by $F \! \adj_{\phi}\!\! G \colon \adjar{B}{A}$ or by 
    $F {\ }_{\eps}^{\text{\tiny{$\eta$}}}\!\!\!\!\adj G \colon \adjar{B}{A}$,
    sometimes even just $F \adj G$, depending on the context.
\end{remark}

\begin{remark}[{proven for example in \cite[{Chapter 4.5}]{riehl}}]
    Let $F \adj G$ be an adjunction.
    Then
    \begin{enumerate}
        \item G preserves limits
        \item F preserves colimits.
    \end{enumerate}
\end{remark}

\section{Examples}
Examples for adjunctions can be found all over mathematics.
Here are a few: \todo{finish}

\begin{example}[Coproduct $\adj$ $\Delta$ $\adj$ Product]
    TODO
\end{example}
\begin{example}[free-forgetful adjunction]
    TODO free group
\end{example}
\begin{example}[Tensor-Hom-Adjunction]
    There is a natural bijection
    \[
        \Hom[A](M \tensor_A N,P) \cong \Hom[A](M,\Hom[A](N,P))
    \]
    This implies that the tensor-product is right-exact, since it preserves cokernels. 
\end{example}
\begin{example}[Galois connection]
    monotone and antitone galois connections, examples are:
    \begin{enumerate}
        \item (Convex sets): TODO
        \item (Fundamental theorem of Galois theory): 
        TODO
        \item  (Algebraic geometry):
        TODO
    \end{enumerate}

\end{example}