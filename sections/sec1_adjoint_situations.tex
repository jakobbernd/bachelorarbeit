
\chapter{Adjoint situations}

\todo{einleitender Satz}
Moreover, adjunctions provide us with many (technically even all) 
examples of monads and comonads, as we will later see.
\begin{proposition}
    Given two functors
    \begin{tikzcd}
        \cat{A} \ar[r,"G",shift left = .60ex]
          & 
        \cat{B} \ar[l,"F",shift left = .60ex]
    \end{tikzcd}
    the following are equivalent: 
    \begin{enumerate}[(a)]
        \item $\exists \eta\colon \id_B \Rightarrow GF$ and $\eps\colon FG \Rightarrow \id_A$ 
        natural transformations such that $\forall a \in Ob(A), b \in Ob(B)$ 
        the following two diagrams commute:
        \begin{equation}\label{eq:triangle identity of adjunction}
            \begin{tikzcd}\tag{triangle identity}
                F(b) \arrow[rd,"\id_{F(b)}"'] \arrow[r, "F(\eta_b)"] & FGF(b) \arrow[d,"\eps_{F(b)}"] \\
                                            & F(b)
            \end{tikzcd}
            \qqq
            \begin{tikzcd} 
        G(a) \arrow[rd,"id_{G(a)}"'] \arrow[r, "\eta_{G(a)}"] & GFG(a) \arrow[d,"G(\eps_a)"] \\
                                            & G(a)
            \end{tikzcd}
        \end{equation}
        \item $\forall a \in Ob(A), b \in Ob(B)$ there is a bijection 
        \[
        \phi_{a,b}\colon\Hom[A](F(b),a)\to \Hom[B](b,G(a))
        \]
        which is natural in a and b, which means that for $p\colon a\to a':$
        \[
            \begin{tikzcd}
                \Hom[A](F(b),a) \arrow[r,"\phi_{a,b}"] \arrow[d,"p \circ \_"] 
                  & \Hom[B](b,G(a)) \arrow[d,"G(p)\circ \_"] \\
                \Hom[A](F(b),a') \arrow[r,"\phi_{a',b}"]
                  & \Hom[B](b,G(a'))
            \end{tikzcd}
        \]
        and for $q\colon b \to b':$
        \[
            \begin{tikzcd}
                \Hom[A](F(b'),a) \arrow[r,"\phi_{a,b'}"] \arrow[d,"\_ \circ F(q)"] 
                  & \Hom[B](b',G(a)) \arrow[d,"\_ \circ q"] \\
                \Hom[A](F(b),a) \arrow[r,"\phi_{a,b}"]
                  & \Hom[B](b,G(a))
            \end{tikzcd}
        \]
    commute.
    \end{enumerate}
\end{proposition}
\begin{bigproof}
    $(a)\implies (b):$ \\
    define 
    \[
        \phi_{a,b}\colon \Hom[A](F(b),a)\to \Hom[B](b,G(a))
    \] by 
    \[
       \phi_{a,b}(g) = G(g) \circ \eta_b 
       \colon b \to G(a)
    \]
    for $g\colon F(b)\to a$ and define 
    \[
      \psi_{a,b} \colon \Hom[B](b,G(a)) \to \Hom[A](F(b),a)
    \] by
    \[
       \psi_{a,b}(f) = \eps_a \circ F(f) 
       \colon F(b) \to a
    \]
    for $f\colon b \to G(a).$
\begin{claim}
$\psi \circ \phi = id$
\end{claim}
\begin{smallproof}
Let $f \colon b \to G(a)$.
\begin{align*}
    \phi(\psi(f)) &= \phi(\eps_a \circ F(f)) \tag{Definition of $\psi$} \\
    &= G(\eps_a \circ F(f)) \circ \eta_b \tag{Definition of $\phi$}\\
    &= G(\eps_a) \circ G(F(f)) \circ \eta_b \tag{Functoriality of $G$}\\
    &= G(\eps_a) \circ \eta_{G(a)} \circ f \tag{Naturality of $\eta$} \\
    &= \id_{G(a)} \circ f \tag{right~\ref{eq:triangle identity of adjunction}} \\
    &= f
\end{align*}
\end{smallproof}
\end{bigproof}

\begin{remark}
    Let $F \adj G$ be an adjoint situation, i.e. 
    $F$ is left-adjoint to $G$ and $G$ is right-adjoint to $F$.
    Then
    \begin{enumerate}
        \item G preserves limits
        \item F preserves colimits.
    \end{enumerate}
\end{remark}

\begin{example}[Galois connection]\todo{those are DUAL adjunctions!}
    blablabla examples include:
    \begin{enumerate}
        \item (Fundamental theorem of Galois theory): 
        this example.
        \item  (Algebraic geometry):
        that example.
    \end{enumerate}

\end{example}
\begin{example}[Coproduct $\adj$ $\Delta$ $\adj$ Product]
    this.
\end{example}
\begin{example}[free-forgetful adjunction]
    that.
\end{example}
\begin{example}[Tensor-Hom-Adjunction]
    There is a natural bijection
    \[
        \Hom[A](M \tensor_A N,P) \cong \Hom[A](M,\Hom[A](N,P))
    \]
    This implies that the tensor-product is right-exact, since it preserves cokernels.
\end{example}
