\chapter{Monads and Comonads}
\section{Definition of Monads and Comonads}
A central notion in algebra is that of a \textit{monoid},
that is, a set $M$ equipped with a map 
\todo{definitionen nicht nummerieren?}
$\mu \colon M \times M \to M$; $(a,b) \mapsto a \cdot b$ 
(often called \textit{multiplication}) and an element $e \in M$
such that the following two axioms hold:
\begin{align}
    \label{eq: associativity for a monoid}  \tag{associativity} 
    (a \cdot b) \cdot c = a \cdot (b \cdot c) 
    \quad &\text{for all} \ a,b,c \in M. \\
    \label{eq: identity element for a monoid} \tag{identity element}
    e \cdot a = a \cdot e = a \quad &\text{for all} \ a \in M 
\end{align}
We can give an equivalent definition in terms of maps and commuting diagrams as follows:
A \textit{monoid} is a set $M$ together with two functions 
\[
    \mu \colon M \times M \to M, \quad 
    e \colon \pt \to M
\]
such that the following diagrams commute: \\

\begin{figure}[H]
    \centering
    \begin{subfigure}{0.4\textwidth}
    \centering
    %\phantomsection\label{name1}
        \begin{tikzcd}
            M \times M \times M \ar[r,"\id \times \mu"] \ar[d,"\mu \times \id"] 
              & M \times M \ar[d,"\mu"] \\
            M \times M \ar[r,"\mu"]
              & M
        \end{tikzcd}
    %\caption*{(cap1)}
    \end{subfigure}
    \hspace{2em}
    \begin{subfigure}{0.4\textwidth}
    \centering
    %\phantomsection\label{name2}
        \begin{tikzcd}
            \pt \times M \ar[rd,"l"'] \ar[r, "e \times \id"] 
            & M \times M \ar[d,"\mu"] 
            & M \times \pt \ar[l,"\id \times e"'] \ar[ld,"r"]\\
            & M
        \end{tikzcd}
    %\caption*{(cap2)}
    \end{subfigure}
    \end{figure}


where $\id$ is the identity on m, and $l$ and $r$ are the canonical bijections
\begin{align*}
    &l \colon \pt \times M \to M;\ l(\ast,m) = m \\
    &r \colon M \times \pt \to M;\ r(m,\ast) = m.
\end{align*}

Explicitly, the first diagram means that for all $a,b,c \in M$:
\[
    (a \cdot b) \cdot c = a \cdot (b \cdot c) 
    \quad \text{for all} \ a,b,c \in M.
\]
which is verbatim the~\ref{eq: associativity for a monoid} axiom, the second diagram means that for all $m \in M$:
\[
  e(\ast) \cdot m = l(\ast,m) = m = r(m,\ast) = m \cdot e(\ast)  
\]
which is clearly the~\ref{eq: identity element for a monoid} axiom 
for the element $e(\ast)$.
This motivates the following definition: \todo{monoid/monad/ monoid object}
\begin{definition}[Monad]
A \textit{Monad} $(T,\mu, \eta) $ in a category $\mathcal{X}$ consists of
\begin{itemize}
    \item an endofunctor $T\colon \mathcal{X} \to \mathcal{X}$
    \item a natural transformation $\eta \colon \id_\mathcal{X} \Rightarrow T$ 
    \item a natural transformation $\mu\colon T^2 \Rightarrow T $
\end{itemize}  
such that the following diagrams commute: \\
% der ' nach dem Text ändert die Position der Pfeil-Beschriftung
\begin{figure}[H]
    \centering
    \begin{subfigure}{0.3\textwidth}
        \centering
        \phantomsection\label{dia: associativity}
        % Content of the first subfigure
        \begin{tikzcd}
            T^3 \ar[r,"T\mu",Rightarrow] \ar[d,"\mu T"',Rightarrow] 
            & T^2 \ar[d,"\mu",Rightarrow] \\
            T^2 \ar[r,"\mu",Rightarrow]
            & T
        \end{tikzcd}
        \caption*{(associativity)}
    \end{subfigure}
    \hspace{2em}
    \begin{subfigure}{0.3\textwidth}
        \centering
        \phantomsection\label{dia: unitality}
        \begin{tikzcd}
            T \ar[rd,"\id_T"',Rightarrow] \ar[r, "\eta T",Rightarrow] 
        & T^2 \ar[d,"\mu",Rightarrow] 
        & T \ar[l,"T \eta"',Rightarrow] \ar[ld,"\id_T",Rightarrow]\\
        & T
        \end{tikzcd}
        \caption*{(unitality)}
    \end{subfigure} 
\end{figure}

In terms of components,~\refassociativity and~\refunitality mean that for every object $x$ of $\mathcal{X}$
the following diagrams commute:

\begin{figure}[H]
\centering
\begin{subfigure}{0.4\textwidth}
\centering
%\phantomsection\label{name1}
\begin{tikzcd}
    T(T(Tx)) \ar[r,"T(\mu_x)"] \ar[d,"\mu_{Tx}"'] 
    & T(Tx) \ar[d,"\mu_x"] \\
    T(Tx) \ar[r,"\mu_x"]
    & Tx
\end{tikzcd}
\caption*{(associativity)}
\end{subfigure}
\hspace{2em}
\begin{subfigure}{0.4\textwidth}
\centering
%\phantomsection\label{name2}
\begin{tikzcd}
    Tx \ar[rd,"\id_{Tx}"'] \ar[r, "\eta_{Tx}"] 
    & T(Tx) \ar[d,"\mu_x"] 
    & Tx \ar[l,"T(\eta_x)"'] \ar[ld,"\id_{Tx}"]\\
    & Tx
\end{tikzcd}
\caption*{(unitality)}
\end{subfigure}
\end{figure}

\end{definition}

\begin{example}[preorder]
Recall: A \textit{preorder} $(\mathcal{P},\le)$ is a category with $\mathcal{P}$ as objects and 
a morphism between $X$ and $Y$ iff $X \le Y$.
A functor $T\colon \mathcal{P} \to \mathcal{P}$ is thus a monotonic function $\mathcal{P}\to \mathcal{P}$
($x \le y \implies Tx\le Ty$).
The existence of the natural transformations $\eta$ is equivalent to
\[x \le Tx \ \forall x \in \mathcal{P}\]
and the existence of $\mu$ is equivalent to
\[T(Tx) \le Tx \ \forall x \in \mathcal{P}\] 
because there is at most one morphism $x \to y$, so the neccessary diagrams commute trivially.\\
Now suppose $\mathcal{P}$ is a \textit{partial order}, i.e. $x \le y \le x \implies x = y \ \forall x,y \in \mathcal{P}$. \\
Then:
\begin{align*}
    x \le Tx \implies Tx \le T(Tx) \\
    T(Tx) \le Tx \implies Tx = T(Tx)
\end{align*}
so a Monad $T$ in a partial order $\mathcal{P}$ is a \textit{closure operation} in $\mathcal{P}$, i.e. 
a monotonic function $T \colon \mathcal{P} \to \mathcal{P}$ 
with $x \le Tx$ and $T(Tx)=Tx \ \forall x \in \mathcal{P}.$ \\
Now every topological space $X$ induces a partial order $\mathcal{P} = (\mathscr{P}(X),\subseteq)$.
Here an example for a closure operation is taking the topological closure $A \mapsto \overline{A}$,
since it holds for all $A \subseteq X$ that $A \subseteq \overline{A}$ and
$\overline{\overline{A}} = \overline{A}$.
\end{example}

\begin{definition}[Comonad]
A \textit{Comonad} $(L,\eps, \omega) $ in a Category $\mathcal{A}$ consists of
\begin{itemize}
    \item an endofunctor $L\colon \mathcal{A} \to \mathcal{A}$
    \item a natural transformation $\eps \colon L \Rightarrow \id_{\mathcal{A}}$ 
    \item a natural transformation $\omega\colon L \Rightarrow L^2 $
\end{itemize}  
such that the following diagrams commute:

\begin{figure}[H]
\centering
\begin{subfigure}{0.4\textwidth}
\centering
%\phantomsection\label{name1}
    \begin{tikzcd}
        L \ar[r,"L\omega",Rightarrow] \ar[d,"\omega L"',Rightarrow] 
            & L^2 \ar[d,"L\omega",Rightarrow] \\
        L^2 \ar[r,"\omega L",Rightarrow]
            & L^3
    \end{tikzcd}
\caption*{(counitality)}
\end{subfigure}
\hspace{2em}
\begin{subfigure}{0.4\textwidth}
\centering
%\phantomsection\label{name2}
    \begin{tikzcd} 
        & L \ar[ld,"\id_L"',Rightarrow] 
        \ar[rd,"\id_L",Rightarrow] \ar[d,"\omega",Rightarrow] & \\
        L 
        & L^2 \ar[l,"\eps L"',Rightarrow] \ar[r,"L \eps",Rightarrow] 
        & L
    \end{tikzcd}
\caption*{(coassociativity)}
\end{subfigure}
\end{figure}


In terms of components, this means that for every object $x$ of $\mathcal{A}$
the following diagrams commute:

\begin{figure}[H]
\centering
\begin{subfigure}{0.4\textwidth}
\centering
%\phantomsection\label{name1}
    \begin{tikzcd}
        Lx \ar[r,"L(\omega_x)"] \ar[d,"\omega_{Lx}"'] 
            & L(Lx) \ar[d,"L(\omega_x)"] \\
        L(Lx) \ar[r,"\omega_{Lx}"]
            & L(L(Lx))
    \end{tikzcd}
\caption*{(counitality)}
\end{subfigure}
\hspace{2em}
\begin{subfigure}{0.4\textwidth}
\centering
%\phantomsection\label{name2}
    \begin{tikzcd}
        & Lx \ar[ld,"\id_{Lx}"'] 
        \ar[rd,"\id_{Lx}"] \ar[d,"\omega_x"] & \\
        Lx 
        & L(Lx) \ar[l,"\eps_{Lx}"'] \ar[r,"L(\eps_x)"] 
        & Lx
    \end{tikzcd}
\caption*{(coassociativity)}
\end{subfigure}
\end{figure}

\end{definition}

\begin{lemma}
    For every object $x$ in $\mathcal{X}$, the following diagram commutes:
    \[
      \begin{tikzcd}
        T(Tx) \arrow[r,"T(\delta_x)"] \arrow[d,"\delta_{Tx}"] 
            & T(T'x) \arrow[d,"\delta_{T'x}"] \\
          T(T'x) \arrow[r,"T'(\delta_x)"]
            & T'(T'x)
      \end{tikzcd}
    \]
    this means \[
        \delta T' \circ T \delta = T' \delta \circ \delta T
        \colon T^2 \Rightarrow (T')^2.
    \]
\end{lemma}
\begin{beweis}
    $\delta_x \colon Tx \to T'x$ is a ring homomorphism.
    Since $\delta \colon T \Rightarrow T'$ is natural transformation, the square commutes.
\end{beweis}
\begin{definition}[Morphism of monads]
    Let $\mathcal{X}$ be a category, let $(T,\eta,\mu)$ and $(T',\eta',\mu')$ be monads in $\mathcal{X}$.
    We say that a natural transformation $\nat[\delta]{T}{T'}$ is a \textit{morphism of monads} if it preserves
    the unit and the multiplication, i.e. the following diagrams commute:

    \begin{figure}[H]
    \centering
    \begin{subfigure}{0.4\textwidth}
    \centering
    %\phantomsection\label{name1}
    \begin{tikzcd}
        \id_T \ar[rd,"\eta'"',Rightarrow] \ar[r, "\eta",Rightarrow] 
        & T \ar[d,"\delta",Rightarrow] \\
        & T'
    \end{tikzcd}
    \caption*{(unit-preserving)}
    \end{subfigure}
    \hspace{2em}
    \begin{subfigure}{0.4\textwidth}
    \centering
    %\phantomsection\label{name2}
    \begin{tikzcd}
        T^2 \ar[r,"\mu",Rightarrow] \ar[d,"\delta T' \circ T\delta"',Rightarrow] 
        & T \ar[d,"\delta",Rightarrow] \\
        T'^2 \ar[r,"\mu'",Rightarrow]
        & T'
    \end{tikzcd}
    \caption*{(multiplication-preserving)}
    \end{subfigure}
    \end{figure}

    %vadjust allows todo in math-mode   
\end{definition}
\begin{definition}[Morphism of comonads]
    Let $\mathcal{A}$ be a category, let $(L,\eps,\omega)$ and $(L',\eps',\omega')$ be comonads in $\mathcal{A}$.
    We say that a natural transformation $\nat[\delta]{L}{L'}$ is a \textit{morphism of monads} if it preserves
    the counit and the comultiplication, i.e. the following diagrams commute:
    \begin{figure}[H]
    \centering
    \begin{subfigure}{0.4\textwidth}
    \centering
    %\phantomsection\label{name1}
    \begin{tikzcd}
        L \ar[r,"\delta",Rightarrow] \ar[rd,"\eps"',Rightarrow]
        & L' \ar[d,"\eps'",Rightarrow] \\
        & \id_A 
    \end{tikzcd}
    \caption*{(counit-preserving)}
    \end{subfigure}
    \hspace{2em}
    \begin{subfigure}{0.4\textwidth}
    \centering
    %\phantomsection\label{name2}
    \begin{tikzcd}
        L \arrow[r,"\omega",Rightarrow] \arrow[d,"\delta",Rightarrow] 
          & L^2 \arrow[d,"\delta L' \circ L\delta",Rightarrow] \\
        L' \arrow[r,"\omega'",Rightarrow]
          & {L'}^2
    \end{tikzcd}
    \caption*{(comultiplication-preserving)}
    \end{subfigure}
    \end{figure}
\end{definition}

\section{The Eilenberg-Moore-Category of a Monad}
This section will answer the question, whether every Monad is induced by an adjunction.

\begin{definition}[Eilenberg-Moore-Category]
    Let $T = (T,\eta,\mu)$ be a monad in a category $\cat{X}$.
    A \textit{$T$-algebra} is a pair $(x,h)$ where $x$ is an object of $\cat{X}$ and $h \colon Tx \to x$ is 
    an arrow such that the following diagrams commute:
    \begin{figure}[H]
    \centering
    \begin{subfigure}{0.4\textwidth}
    \centering
    %\phantomsection\label{name1}
    \begin{tikzcd}
        T^2x \arrow[r,"Th"] \arrow[d,"\mu"] 
          & Tx \arrow[d,"h"] \\
        T \arrow[r,"h"]
          & x
    \end{tikzcd}
    %\caption*{(cap1)}
    \end{subfigure}
    \hspace{2em}
    \begin{subfigure}{0.4\textwidth}
    \centering
    %\phantomsection\label{name2}
    \begin{tikzcd}
        x \ar[r,"\eta_x"] \ar[rd,"id_x"']
        & Tx \ar[d,"h"] \\
        & x 
    \end{tikzcd}
    %\caption*{(cap2)}
    \end{subfigure}
    \end{figure}
    We call $h$ the \textit{stucture map} of $(x,h)$.
    A \textit{morphism of $T$-algebras} $f \colon (x,h) \to (x',h')$ is an arrow
    $f \colon x \to x'$ such that
    \[
        \begin{tikzcd}
            Tx \arrow[r,"Tf"] \arrow[d,"h"] 
            & Tx' \arrow[d,"h'"] \\
            x \arrow[r,"f"]
            & x'
        \end{tikzcd}
    \]
    commutes.
    The set of all $T$-algebras together with their morphisms form a category,
    which is called the \textit{Eilenberg-Moore-Category} and denoted by $\cat{X}^T$.
\end{definition}
\todo{Proof that this is indeed a category?}
\begin{theorem}[Every monad is defined by its T-algebras]
    Let $(T,\eta,\mu)$ be a monad in a category $\cat{X}$.
    Then there is an adjunction $F^T \adj G^T$, where $F^T$ and $G^T$ are functors
    \begin{tikzcd}
        \cat{X^T} \ar[r,"G^T",shift left = .60ex]
          & 
        \cat{X} \ar[l,"F^T",shift left = .60ex]
    \end{tikzcd}
    such that the monad induced by this adjunction is $(T,\eta,\mu)$.
\end{theorem}
\begin{beweis}
Define $F^T \colon \cat{X} \to \cat{X}^T$ by
\[
    \begin{tikzcd}
        x \arrow[r,mapsto] \arrow[d,"f"]
          & (Tx,\mu_x) \arrow[d,"Tf"] \\
        x' \arrow[r,mapsto]
          & (Tx',\mu_{x'})
    \end{tikzcd}
\]
$(Tx,\mu_x)$ is indeed a $T$-algebra, since $\mu_x$ is an arrow $T^2x \to Tx$
and the diagrams 
\begin{figure}[H]
\centering
\begin{subfigure}{0.4\textwidth}
\centering
%\phantomsection\label{name1}
\begin{tikzcd}
    T^3x \arrow[r,"T(\mu_x)"] \arrow[d,"\mu_{Tx}"] 
      & T^2x \arrow[d,"\mu_x"] \\
    T^2x \arrow[r,"\mu_x"]
      & Tx
\end{tikzcd}
%\caption*{(cap1)}
\end{subfigure}
\hspace{2em}
\begin{subfigure}{0.4\textwidth}
\centering
%\phantomsection\label{name2}
\begin{tikzcd}
    Tx \ar[r,"\eta_{Tx}"] \ar[rd,"id_{Tx}"']
    & T^x \ar[d,"\mu_x"] \\
    & Tx 
\end{tikzcd}
%\caption*{(cap2)}
\end{subfigure}
\end{figure}
are just the commuting diagrams for the~\refassociativity
respectively left~\refunitality axioms from the definition of a Monad.

$Tf \colon (Tx,\mu_x) \to (Tx',\mu_{x'})$ is indeed a morphism of $T$-algebras,
since the commutativity of 
\[
  \begin{tikzcd}
      T^2x \arrow[r,"T^2(f)"] \arrow[d,"\mu_x"] 
        & T^2x' \arrow[d,"\mu_{x'}"] \\
      Tx \arrow[r,"T(f)"]
        & Tx'
  \end{tikzcd}  
\]
is given by naturality of $\mu$. The functoriality of $F^T$ follows from the functoriality
of $T$.

Define $G^T \colon \cat{X^T} \to \cat{X}$ by
\[
    \begin{tikzcd}
        (x,h) \arrow[r,mapsto] \arrow[d,"f"] 
          & x \arrow[d,"f"] \\
        (x',h') \arrow[r,mapsto]
          & x'
    \end{tikzcd}   
\]
so $G$ is just the forgetful functor.
\begin{claim*}
    $G^T \circ F^T = T$ and $F^TG^T(x,h) = (Tx,\mu_x$).
\end{claim*}
\begin{proof}[Proof of claim]
    Let $x \in \cat{X}.$ Then $G^T(F^T(x)) = G^T(Tx,\mu_x) = Tx.$
    Now let $f \colon x \to y$. Then $G^T(F^T(f))=G^t(Tf)=Tf.$
    Finally, $F^TG^T(x,h) = F^T(x) = (Tx,\mu_x).$
\end{proof}
So we can set 
\[
  \eta^T := \eta \colon \id_{\cat{X}} \Rightarrow G^TF^T 
\]
as the unit and we can define $\eps^T \colon F^TG^T \to \id_{\cat{X}^T}$ by
\[
    \eps^T_{(x,h)} := h \colon (Tx,\mu_x) \to (x,h).
\]
$h$ is a morphism of $T$-algebras because $(x,h)$ is a $T$-algebra, since both statements
mean that the diagram 
\[
    \begin{tikzcd}
        T^2x \arrow[r,"Th"] \arrow[d,"\mu"] 
          & Tx \arrow[d,"h"] \\
        T \arrow[r,"h"]
          & x
    \end{tikzcd}
\]
commutes. $\eps^T$ is natural, because if $f \colon (x,h) \to (x',h')$
is a morphism of $T$-algebras, naturality means that the diagram
\[
    \begin{tikzcd}
        Tx \arrow[r,"Tf"] \arrow[d,"h"]
          & Tx' \arrow[d,"h'"] \\
        x \arrow[r,"f"]
          & x'
    \end{tikzcd}
\]
but this is exactly the definition of $f$ being a morphism of $T$-algebras.
\todo{triangle identies and induces T}
\end{beweis}
\begin{theorem}[Comparison of adjunctions with algebras]

\end{theorem}
\section{The Kleisli category of a Monad}
There is another way to induce a Monad by an adjunction:
\begin{definition}[Kleisli category]
    Let $\cat{X}$ be a category, $T = (T,\eta, \mu)$ be a monad in $\cat{X}$.
    The \textit{Kleisli category $\cat{X}_T$} is defined by
    \begin{itemize}
        \item objects the same as in $\cat{X}$, but we relabel $x$ to $x_T$ for all $x \in \cat{X}$.
        \item for $x_T, y_T \in \cat{X}_T$, $f\colon x \to Ty$ is a morphism which we
        denote by $f^b \colon x_T \to y_T$.
        \item composition will be denoted by $\bullet$ for distinction and is defined by
        \[
            g^b \bullet f^b := (\mu_z \circ Tg \circ f)^b \colon x_T \to z_T
        \]
        for $f^b \colon x_T \to y_T$, $g^b \colon y_T \to z_T$. 
        This is indeed again a morphism:
        \begin{tikzcd}
            x \arrow[r,"f"]
              & Ty \arrow[r,"Tg"]
                & T^2z \arrow[r,"\mu_z"]
                  & Tz
        \end{tikzcd}
    \end{itemize}
\begin{claim*}
    This defines a category.
\end{claim*}
\begin{proof}[Proof of claim]
\underline{associativity}: Let
\begin{tikzcd}
    x_T \arrow[r,"f^b"]
      & y_T \arrow[r,"g^b"]
        & z_T \arrow[r,"h^b"]
          & w_T
\end{tikzcd}
be objects and morphisms in the Kleisli category.
\begin{align*}
    (h^b \bullet g^b) \bullet f^b &= (\mu_w \circ Th \circ g)^b \bullet f^b \\
    &= (\mu_w \circ T(\mu_w \circ Th \circ g) \circ f)^b \\
    &= (\mu_w \circ T\mu_w \circ T^2h \circ Tg \circ f)^b.
\end{align*}
Now the~\refassociativity axiom for the Monad T states that
\[
    \begin{tikzcd}
        T(T(Tw)) \ar[r,"T(\mu_w)"] \ar[d,"\mu_{Tw}"'] 
        & T(Tw) \ar[d,"\mu_w"] \\
        T(Tw) \ar[r,"\mu_w"]
        & Tw 
    \end{tikzcd}
\]
commutes, hence 
\[
    (\mu_w \circ T\mu_w \circ T^2h \circ Tg \circ f)^b
    = (\mu_w \circ \mu_{Tw} \circ T^2h \circ Tg \circ f)^b
\]
By naturality of $\mu$, the diagram
\[
    \begin{tikzcd}
        T^2z \arrow[r] \arrow[d] 
          & T^3w \arrow[d] \\
        Tz \arrow[r]
          & T^2w
    \end{tikzcd}
\]
commutes, so it follows that
\begin{align*}
    (\mu_w \circ \mu_{Tw} \circ T^2h \circ Tg \circ f)^b
    &= (\mu_w \circ Th \circ \mu_z \circ Tg \circ f)^b \\
    &= h^b \bullet (g^b \bullet f^b) \\
\end{align*}
\underline{identity axiom}: Let $f^b \colon x_T \to y_T$ be a morphism.
\begin{align*}
    f^b \bullet (\eta_x)^b = (\mu_x \circ Tf \circ \eta_x)^b
    = (\mu_x \circ \eta_{Ty} \circ f)^b
    = (\id_{Ty} \circ f)^b = f^b
\end{align*}
where the second equality follows from the naturality of $\eta$ and 
the third equality is due to the left~\refunitality law for T.
\begin{align*}
    (\eta_y)^b \bullet f^b = (\mu_y \circ T\eta_y \circ f)^b
    =(id_{Ty}\circ f)^b = f^b
\end{align*}
where the second equality is due to the right~\refunitality law for T.
This proves that for $x_T \in \cat{X}_T$ we have $\id_{x_T} = (\eta_x)^b
\in \Hom[\cat{X}_T](x_T,x_T)$
\end{proof}
\begin{theorem}
    There is an adjoint situation $F_T \adj G_T \colon$ 
    \begin{tikzcd}
        \cat{X_T} \ar[r,,shift left = .60ex]
          & 
        \cat{X} \ar[l,,shift left = .60ex]
    \end{tikzcd}
    such that $T$ is the induced monad by this adjunction.
\end{theorem}
\end{definition}