\chapter{Monads and Comonads}
\section{Definition of Monads and Comonads}
\begin{definition}[Monad]
    A \textit{Monad} $(T,\mu, \eta) $ in a Category $\mathcal{X}$ consists of
    \begin{itemize}
        \item an endofunctor $T\colon \mathcal{X} \to \mathcal{X}$
        \item a natural transformation $\eta \colon \id_\mathcal{X} \implies T$ 
        \item a natural transformation $\mu\colon T^2 \implies T $
    \end{itemize}  
    such that the following diagrams commute:
    \begin{enumerate}[(a)]
    \item 
    \begin{minipage}{0.4\linewidth}
    \centering
    \begin{tikzcd}
        T^3 \arrow[r] \arrow[d] 
          & T^2 \arrow[d] \\
        T^2 \arrow[r]
          & T
    \end{tikzcd}
    \end{minipage}
    $\mathrm{and}$
    \begin{minipage}{0.4\linewidth}
    \centering
    \begin{tikzcd}
        T \arrow[rd,"\id_T"'] \arrow[r, "\eta T"] 
        & T^2 \arrow[d,"\mu"] 
        & T \arrow[l,"T \eta "'] \arrow[ld,"\id_T"]\\
        & T
    \end{tikzcd}
    \end{minipage}
    % der ' nach dem Text ändert die Position der Pfeil-Beschriftung
\end{enumerate}
\end{definition}

\begin{example}[preorder]
    Recall: A \textit{preorder} $(\mathcal{P},\le)$ is a category with $\mathcal{P}$ as objects and 
    a morphism between $X$ and $Y$ iff $X \le Y$.
A functor $T\colon \mathcal{P} \to \mathcal{P}$ is thus a monotonic function $\mathcal{P}\to \mathcal{P}$
($x \le y \implies Tx\le Ty$).
The existence of the natural transformations $\eta$ is equivalent to
\[x \le Tx \ \forall x \in \mathcal{P}\]
and the existence of $\mu$ is equivalent to
\[T(Tx) \le Tx \ \forall x \in \mathcal{P}\] 
because there is at most one morphism $x \to y$, so the neccessary diagrams commute trivially.\\
Now suppose $\mathcal{P}$ is a \textit{partial order}, i.e. $x \le y \le x \implies x = y \ \forall x,y \in \mathcal{P}$. \\
Then:
\begin{align*}
    x \le Tx \implies Tx \le T(Tx) \\
    T(Tx) \le Tx \implies Tx = T(Tx)
\end{align*}
so a Monad $T$ in a partial order $\mathcal{P}$ is a \textit{closure operation} in $\mathcal{P}$, i.e. 
a monotonic function $T \colon \mathcal{P} \to \mathcal{P}$ 
with $x \le Tx$ and $T(Tx)=Tx \ \forall x \in \mathcal{P}.$ \\
Now every topological space $X$ induces a partial order $\mathcal{P} = (\mathscr{P}(X),\subseteq)$.
Here an example for a closure operation is taking the topological closure $A \mapsto \overline{A}$,
since it holds for all $A \subseteq X$ that $A \subseteq \overline{A}$ and
$\overline{\overline{A}} = \overline{A}$.
\end{example}

\begin{definition}[Comonad]
    A \textit{Comonad} $(L,\eps, \omega) $ in a Category $\mathcal{A}$ consists of
    \begin{itemize}
        \item an endofunctor $L\colon \mathcal{A} \to \mathcal{A}$
        \item a natural transformation $\eps \colon L \implies \id_{\mathcal{A}}$ 
        \item a natural transformation $\omega\colon L \implies L^2 $
    \end{itemize}  
    such that the following diagrams commute:
    \begin{enumerate}[(a)]
    \item 
    \begin{minipage}{0.4\linewidth}
    \centering
    \begin{tikzcd}
        L \arrow[r,"L\omega"] \arrow[d,"\omega L"'] 
          & L^2 \arrow[d,"L\omega"] \\
        L^2 \arrow[r,"\omega L"]
          & L^3
    \end{tikzcd}
    \end{minipage}
    $\mathrm{and}$
    \begin{minipage}{0.4\linewidth}
    \centering
    \begin{tikzcd}
        L   
        & L^2 \arrow[l, "\eps L"'] \arrow[r,"L \eps "]  
        & L  \\
        & L \arrow[lu,"\id_L"] \arrow[u,"\omega"'] \arrow[ru,"\id_L"']
    \end{tikzcd}
    \end{minipage}
    % der ' nach dem Text ändert die Position der Pfeil-Beschriftung
\end{enumerate}
\end{definition}


\begin{definition}[Morphism of monads]
    Let $\mathcal{X}$ be a category, let $(T,\eta,\mu)$ and $(T',\eta',\mu')$ be monads in $\mathcal{X}$.
    We say that a natural transformation $\nat[\delta]{T}{T'}$ is a \textit{morphism of monads} if it preserves
    the unit and the multiplication, i.e. the following diagrams commute:
    \[\begin{tikzcd}
        \id_x \arrow[rd,"\eta'_x"'] \arrow[r, "\eta_x"] & Tx \arrow[d,"\delta_x"] \\
                                    & T'x
    \end{tikzcd}
        \]
    \[\begin{tikzcd}
        T^2x \arrow[r,"\mu_x"] \arrow[d,"\delta T' \circ T\delta"'] 
          & Tx \arrow[d,"\delta_x"] \\
        T'^2 \arrow[r,"\mu'_x"]
          & T'x
    \end{tikzcd}
    \vadjust{\todo{show that the other composition is the same(siehe iPad)}} 
    %vadjust allows todo in math-mode
    \]   
\end{definition}
\begin{definition}[Morphism of comonas]
    \todo{definition}
\end{definition}
\end{document}